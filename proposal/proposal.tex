\documentclass[11pt]{article}

\title{Theatre Science (Proposal to Stage Committee)}
\author{Ashley Williamson and William Denton}
\date{03 July 2019}

%% \usepackage[landscape]{geometry}

\usepackage[print]{booklet}  % final printing run
\setpdftargetpages


\usepackage[T1]{fontenc}
\usepackage[utf8]{inputenc}
\usepackage[english]{babel}

% \usepackage{microtype}
\usepackage[activate={true,nocompatibility},final,tracking=true,kerning=true,spacing=true,factor=1100,stretch=10,shrink=10]{microtype}
\microtypecontext{spacing=nonfrench}

\usepackage[osf]{Baskervaldx}

\usepackage{enumitem}
\setlist{noitemsep}

\usepackage{graphicx}

\usepackage{ccicons}

% % Turn \url and \href into hyperlinks in PDFs.
% \usepackage[pdfborder={0 0 0},colorlinks=true,urlcolor=blue]{hyperref}
% \urlstyle{same}
% \newcommand{\link}[1]{$<$\url{#1}$>$}

\usepackage{lipsum}

\begin{document}

\begin{center}

  \resizebox{4.5in}{!}{\textsc{williamson \& denton investigate}}

  \vspace{2cm}

  \resizebox{4.5in}{!}{\textsc{theatre}}\\
  \vspace{0.5cm}
  \resizebox{4.5in}{!}{\textsc{science}}\\

  \vspace{3cm}

  \resizebox{4in}{!}{\texttt{Limitation. Algorithm.}}\\
  \vspace{0.2cm}
  \resizebox{4in}{!}{\texttt{Method.  Process. Score.}}\\

\vspace{5cm}

\resizebox{4in}{!}{Proposal to Stage Committee · 03 July 2019}\\

\vfill

\ccby

\texttt{https://github.com/wdenton/theatrescience}

\end{center}

\clearpage

\section{What to expect today}

We will be making a ten-minute presentation while wearing lab coats.  We are going to ask the Stage Committee for two Club Nights so that we can perform a theatre experiment.

% You have this pamphlet in advance.  It explains what we are asking for and why.  There are no surprises.  Everything is planned ahead of time.  Our request embodies the nature and intent of our project.

\section{Why we are here}

Stage productions at the Club are limited by some constraints.  Everyone on the Committee will recognize these:

\begin{itemize}

  \item There is a lack of rehearsal time.
  \item Not everyone attends all rehearsals.
  \item It's hard to learn lines.
  \item It's hard to remember blocking.
  \item Few people can design and build sets.
  \item The lighting grid is fairly basic.

\end{itemize}

But constraints can be helpful to creativity.  What if instead of trying to work around them, we \textit{built a performance on them}?

\begin{itemize}

  \item Have very few rehearsals, if any.
  \item Design flexibility into casting and roles.
  \item Let everyone see their dialogue as needed.
  \item Show the blocking, even by drawing lines on the floor.
  \item Keep the stage simple.  Use the Great Hall itself.
  \item Make the most possible of the lighting grid as is.

\end{itemize}

(Of course there are some things the Club is very good at, like \textit{costumes}.  We'll use them as much as possible.)

\textit{Use the limitations}.  Create algorithms, methods and processes for a performance, and put them into a score.\footnote{Plays have scripts.  Performances have scores.}

\texttt{Limitation.  Algorithm.  Method.  Process. Score.}

We have formed a hypothesis and we wish to perform experiments and analyze the results, which we will then write up and disseminate. \textit{Theatre science}.

\section{What we request}

We ask for two nights to run \textit{Theatre Science} performances.  Ideally these would be Club Nights close together, but one Club night and another special dinner would also work.  (We know how activities are planned.)

We also request a budget of \$150 for printing and small supplies like spike tape and index cards.

\section{What to expect at the show}

Both the stage and the Great Hall will be used as performance space.  Tables will be arranged specially; we imagine seating would be limited to about 60 people (more on the second night).  There will be dialogue and directions projected or posted on the walls and lines marked on the floor.  Sound design will be a key part of the show.  Everyone at dinner will feel they are inside a performance (but there is no forced audience participation).

Scores will be at each place setting.   They will be \textit{complete scores for the entire show}.  Everyone can read what is going to happen and then follow along.  Total transparency: there are no secrets---but there is \textit{chance} and \textit{serendipity}.

\textbf{Beta test (night 1):}  The first night will pre-arranged and cast with experienced theatre people, like you.  We will collect data, analyze the results, and adjust the score.  The performance will be welcoming, engaging, friendly and participatory.  It will be designed so that after seeing what it's like people will \textit{want} to try it out.

\textbf{Production release (night 2):} This will be performed by non-theatre people---people you never see on stage, but who will see the first show and think, ``I could do that. I \textit{want} to do that!''

\section{What is the show about?}

\textbf{Information literacy}, defined as ``abilities encompassing the reflective discovery of information, the understanding of how information is produced and valued, and the use of information in creating new knowledge and participating ethically in communities of learning.''  It will be based on the threshold concepts in the \textit{Framework for Information Literacy for Higher Education}.\footnote{http://www.ala.org/acrl/standards/ilframework}

\textit{Theatre Science} combines form (devised theatre) and content (information literacy) to make a show based on principles that are core to education at all levels and to a civil society.

\section{Why us?}

\textit{Theatre Science} combines the interests and expertise of Ashley Williamson (theatre studies) and William Denton (library science).

Ashley Williamson will have her PhD in theatre studies from the University of Toronto in August.  At the Club she has been involved in many stage performances, including directing the Spring Revue (\textit{Vamp Till Ready}) in 2015 and running Ad Lib for several years.  She also produced the Boar's Head Dinner, which uses a performance score as \textit{Theatre Science} will.

William Denton has a master's in information studies from the University of Toronto and currently works as scholarly analytics librarian at York University.  His online art projects include GHG.EARTH and \textit{Listening to Art}.  At the Club, aside from being Librarian, he has done lighting and sound for a number of stage productions and written for many Revues.

Together, Williamson and Denton have produced several Ad Libs, including two Dada Nights, ``The Dentonian L'', ``Readings and More'' (of which Rosemary Aubert said: ``Friday night was one of the best times I ever had at the Club'') and most recently ``L'Atelier des Refusés.''

Williamson and Denton have a proven track record and brand recognition.

\section{Next steps}

Ad Lib is booked for alpha tests on 18 October and 15 November 2019.  We will be preparing for these over the summer and early fall.

Promotion of the beta test and production night would begin in September in the \textit{LAMPSletter}.  We would actively recruit people to come to all the performances.

After the production, we will move the project beyond the Club,  write up what we did and disseminate the results at conferences and in publications. The combination of form, content and process will be useful and practical to many audiences for many purposes, especially pedagogical.

\section{Open license}

Everything associated with \textit{Theatre Science} will be made available under a Creative Commons Attribution (CC BY) license.  This lets other people adapt, reuse and repurpose the work, as long as they give us credit.

\end{document}
