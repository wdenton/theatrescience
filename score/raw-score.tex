\documentclass[17pt]{extarticle}
% \documentclass[17pt]{memoir}

\title{Theatre Science}
\author{Williamson and Denton}
\date{February 2020\\Basic title page for now}

\usepackage[margin=4cm]{geometry}

\usepackage{extsizes}

% \usepackage{pifont}

\usepackage[activate={true,nocompatibility},final,tracking=true,kerning=true,spacing=true,factor=1100,stretch=10,shrink=10]{microtype}
\microtypecontext{spacing=nonfrench}

\usepackage[T1]{fontenc}

\usepackage[utf8]{inputenc}

\usepackage{xcolor}

\usepackage{amssymb}

\usepackage{fancybox}
\usepackage{dashbox}

\usepackage{enumitem}
\setlist{noitemsep}

\usepackage{marginnote}
\newcommand{\mysidenote}[1]{\marginnote{\raggedrightmarginnote \footnotesize #1}}

% Disable section numbering for the whole document.
\setcounter{secnumdepth}{0}

\usepackage{fancyhdr}
\pagestyle{fancy}
\fancyhead[L]{Theatre Science}

% Baskerville font
\usepackage[osf]{Baskervaldx}

% Use space between paragraphs, with no indent.
\usepackage{parskip}
% No paragraph indent.
% \setlength{\parskip}{\baselineskip}%
% \setlength{\parindent}{0pt}%

\usepackage{ifthen}
% \providecommand\speaker{sone}
% \ifthenelse{ \equal{\blackandwhite}{true} }{
% % "black and white" mode; do something..
% }{
% % "color" mode; do something different..
% }

\newcommand{\blankspace}{\_\_\_\_\_\_\_\_}

\newcommand{\musician}[1]{    \ifthenelse{ \equal{\speaker}{musician}    }{\textbf{MUSICIAN: #1}}{MUSICIAN: #1}}
\newcommand{\primary}[1]{     \ifthenelse{ \equal{\speaker}{primary}     }{\textbf{PRIMARY: #1}}{PRIMARY: #1}}
\newcommand{\secondaryone}[1]{\ifthenelse{ \equal{\speaker}{secondaryone}}{\textbf{SECONDARY 1: #1}}{SECONDARY 1: #1}}
\newcommand{\secondarytwo}[1]{\ifthenelse{ \equal{\speaker}{secondarytwo}}{\textbf{SECONDARY 2: #1}}{SECONDARY 2: #1}}
\newcommand{\secondarythr}[1]{\ifthenelse{ \equal{\speaker}{secondarythr}}{\textbf{SECONDARY 3: #1}}{SECONDARY 3: #1}}
\newcommand{\tertiaryone}[1]{ \ifthenelse{ \equal{\speaker}{tertiaryone} }{\textbf{TERTIARY 1: #1}}{TERTIARY 1: #1}}
\newcommand{\tertiarytwo}[1]{ \ifthenelse{ \equal{\speaker}{tertiarytwo} }{\textbf{TERTIARY 1: #1}}{TERTIARY 2: #1}}
\newcommand{\tertiarythr}[1]{ \ifthenelse{ \equal{\speaker}{tertiarythr} }{\textbf{TERTIARY 1: #1}}{TERTIARY 3: #1}}

% Box formatting, so they have nice lines and are well padded
\setlength{\fboxsep}{1em}
\setlength{\fboxrule}{2pt}

% tsdirectionforall = Direction for All
\newcommand{\tsdirectionforall}[1]{\setlength{\fboxrule}{2pt}\vspace{0.5\baselineskip}\fbox{\parbox{\textwidth}{#1}}\setlength{\fboxrule}{1pt}\vspace{0.5\baselineskip}}

% tstechnical = Technical direction
\newcommand{\tstechnical}[1]{\dbox{\vspace{0.5\baselineskip}\parbox{\textwidth}{\textit{#1}}}\vspace{0.5\baselineskip}}

\begin{document}

\maketitle

\vfill

Details: https://theatrescience.org/ CC BY

(This score is for: \speaker)

\newpage

\tableofcontents

\newpage

\section{Casting}

\begin{description}[align=right,labelwidth=2cm]

  \item [M] one musician
  \item [P] one PRIMARY speaker
  \item [S] three SECONDARY roles
  \begin{description}[align=right,labelwidth=2cm]
    \item [S1] can go up the stairs to the stage; does one Mushroom Verbatim
    \item [S2] always on the floor; does one Mushroom Verbatim
    \item [S3] limited movement; stays in the podium zone
  \end{description}
  \item [T] three TERTIARY roles
  \begin{description}[align=right,labelwidth=2cm]
    \item [T1] limited movement on stage
    \item [T2] limited movement on stage
    \item [T3]  stage right; does not move; can remain seated
  \end{description}
  \item [Q] one technical person in the booth
  \item [R] one problem solver on the floor

\end{description}

PRIMARY does about 45\% of the speaking, while the SECONDARY group shares another 45\% and TERTIARY shares 10\%.

\newpage

\section{Instructions}

Please read this section carefully before performing.  It is short but the instructions are important.

\subsection{Formatting}

Boxes have \textit{instructions}, which are to be followed but not spoken.

\tsdirectionforall{A solid box has directions that \textsc{everyone} should read.}

% \setlength{\fboxrule}{1pt}

\tstechnical{A dashed box has \textit{technical instructions}, which are also in italics to set them apart. Everyone can read them, but only Q and R will act on them.}

\tsdirectionforall{Square brackets are for instructions for \textit{one person}.  Below are two examples.}

\musician{[PLAY A NOTE.]}

\primary{[POINT TO THE TERTIARY GROUP.]}

\tsdirectionforall{Your lines are in bold.  Here is an example of what SECONDARY 1 sees.}

\secondaryone{(\textit{thoughtfully}):  When it is your turn to say something, the line will be bold, like this.}

\secondarytwo{Authority is constructed ...}

\secondarythr{... and contextual.}

\tsdirectionforall{PAUSE 1}

\secondaryone{Once again, bold lines are the ones you say.}

\subsection{Pauses}

[PAUSE 1] means pause for a count of one, which you might time by saying to youself, ``One one thousand.''.  [PAUSE 2] means pause for a count of two, which you might time by saying to yourself, ``One one thousand, two one thousand.''  [PAUSE 5], you can see, would be a long pause where you count up to 5.  Don't rush a pause.  The audience will wait for you.

\subsection{Pronunciation}

Sometimes there will be instructions like \textit{quickly} or \textit{thoughtfully} about how you should say something.  Try your best, but don't worry about it.

Words that might be hard to pronounce, like the name \textit{Schoenberg}, will have a phonetic pronunciation beside them in square brackets: [SHERN-berg].  The capital letters indicate emphasis.  Give it your best shot, but if you make a mistake, don't worry about it, try again or just move on.  If the audience noticed, they will forget quickly .

The John Cage composition \textit{4' 33''} should always be pronounced in full as ``four minutes and thirty-three seconds.''

\subsection{For the musician}

We use the piano to help indicate when a new section begins.  When you see the instruction PLAY A NOTE, do this:  pick any key on the piano (black or white, but not from too high on the right, because those high sounds won't resonate as long), hit it hard, and keep pressing the key down until you can no longer hear any sound from it, even if performers start talking.  When you can't hear the note any more, wait a few more seconds to be sure, then lift your finger up.  Then wait for the next scenario.

\newpage

\section{The score}

\subsection{Setup: technical}

\tstechnical{R: Setup instructions here.  Projector, laptop, screen, etc..}

\tstechnical{Q LX: Lighting setup.  GM \(\uparrow\) 4. }

\tstechnical{Q SOUND: Prepare the thirty-minute introductory background sound with time announcements.}

\tstechnical{Q SOUND: At 1930, play the thirty-minute introductory background.  It will end at 2000.}

\newpage

\subsection{Setup: cast}

\tsdirectionforall{Instructions for the cast: where to stand or sit, etc.}

\newpage

\subsection{Scenario 0: Introductory film (3 min)}

\tstechnical{R: Show the film.  When it is finished, move to an empty black screen.}

\tstechnical{Q SX: Announcement on microphone.  ``Ladies and gentleman, performing Theatre Science tonight are:\\
  \blankspace as the Musician, \\
  \blankspace as Primary, \\
  \blankspace as Secondary One, \\
  \blankspace as Secondary Two, \\
  \blankspace as Secondary Three, \\
  \blankspace as Tertiary One, \\
  \blankspace as Tertiary Two, \\
  \blankspace as Tertiary Three, \\
  William Denton as Q and Ashley Williamson as R. \\
  You may follow along with the score provided.  Theatre Science is now beginning.  Theatre Science has begun.''}

\tstechnical{Q: Hit the gong loudly.}

\newpage

\subsection{Scenario 1: Introduction (4 min)}

\tstechnical{Q: LX: Lights up.}

\tsdirectionforall{MUSICIAN:  Follow the red line until you are behind the piano, then sit on the piano stool. }

\musician{[PLAY A NOTE.]}

\tsdirectionforall{TERTIARY:  Follow the orange line until you come to stand in a square.  T1 and T2: you will stand.  T3: you may sit or stand, as you choose.}

\tsdirectionforall{PRIMARY: Follow the purple line.  Stop on the square.}

\tsdirectionforall{SECONDARY:  S1: follow the blue line on the right and stop on the X.  S2: follow the green line on the left and stop on the $\times$.  S3: follow the pink tape and stop on the chevrons.}

\tsdirectionforall{When you have arrived at your position, wait for the note from the piano to stop.}

\tsdirectionforall{PRIMARY:  When you can no longer hear the piano, wait one more second.  Slowly raise your arms over your head, then slowly lower them back to your sides.}

\tsdirectionforall{SECONDARY and TERTIARY:  Copy Primary's arm movements as best you can.}

\primary{[PAUSE 1] Good evening, everyone.  This show is about a concept from the field of library and information science.  The concept is:  \textit{Authority is constructed and contextual.} }

\tstechnical{R: SLIDE: ``Authority is constructed and contextual.''}

\tsdirectionforall{Turn to the slide being projected and point at the screen.}

\primary{This comes from a field of librarianship called \textit{information literacy}.  First, let's define that.}

\secondaryone{\textit{Literacy}, on its own, can be defined in a very simple way as being able to read and write.  But the world is a lot more complicated now than it was five thousand years ago.  There is writing all around us, on paper or on screens.}

\secondarytwo{There are many different kinds of literacies.  The equivalent with numbers is \textit{numeracy}.  Being able to read and follow maps is a kind of literacy. There is also \textit{digital literacy}, about being able to use digital tools, and \textit{media literacy}. }

\secondarythr{For librarians the key concept is \textit{information literacy}.  Here's a definition.  It's dense, but we'll break it down.}

\tstechnical{R: SLIDE: ``Information literacy is the set of integrated abilities encompassing the reflective discovery of information, the understanding of how information is produced and valued, and the use of information in creating new knowledge and participating ethically in communities of learning.''}\footnote{Citation needed.}

\tsdirectionforall{While PRIMARY reads: SECONDARY 1, follow the blue line until you are standing on the triangle.  SECONDARY 2, follow the green line until you are at the ?????.}

\primary{[POINT BRIEFLY TO THE SCREEN, THEN READ.]  ``Information literacy is the set of integrated abilities encompassing the reflective discovery of information, the understanding of how information is produced and valued, and the use of information in creating new knowledge and participating ethically in communities of learning.''  Put more simply, information literacy is the ability to do these with information:}

\secondaryone{Find.}

\tertiaryone{Use.}

\secondarytwo{Understand.}

\tertiarytwo{Evaluate.}

\secondarythr{Integrate.}

\tertiarythr{Share.}

\primary{They don't have to all be done at the same time, or in that order.}

\secondaryone{Just reading the news requires you to understand, evaluate and integrate, but the news is coming at you all the time, you don't have to go out and find it.}

\secondarytwo{And it happens in different contexts.  Helping a kid in elementary school with a homework assignment about trees is a lot easier than understanding changing provincial policy about the Green Belt, but the steps are the same.}

\secondarythr{A lot depends on people.  Information resources don't have to be written.  If they are written, they're written by people.  But sometimes they just actually are people.  We turn to others for information or advice or expertise.}

\primary{This definition of information literacy on the screen comes from a division of the American Library Association that is a professional association for academic librarians.  One of the things it works on is information literacy.  In 2016, it published a short document called \textit{Framework for Information Literacy for Higher Education}.}

% \tstechnical{R: SLIDE: Framework for Information Literacy in Higher Education banner.}

\primary{Librarians used to think of information literacy as being very tied to information technology.  It was mostly about using computers to do research.  And they had long lists of specific tasks that information literate people should be able to do, like ``Record all pertinent citation information for future reference.''  That \textit{is} important, but as the librarians taught information literacy to students in colleges and universities, and as the world changed, they found this long list of precisely defined skills wasn't the best for:}

\secondaryone{Find.}

\tertiaryone{Use.}

\secondarytwo{Understand.}

\tertiarytwo{Evaluate.}

\secondarythr{Integrate.}

\tertiarythr{Share.}

\primary{So they designed a ``framework'' that has six ``frames.''  These frames are ``threshold concepts.''  That's when you go from just having a set of facts or rules to understanding the larger picture of how it all fits together.  When you hit that point, you have a deeper understanding of the subject, but you also know things are more complicated than they used to seem.  The six frames are these.}

\secondaryone{Authority is constructed and contextual.}

\secondarytwo{Information creation as a process.}

\secondarythr{Information has value.}

\secondaryone{Research as inquiry.}

\secondarytwo{Scholarship as conversation.}

\secondarythr{Searching as strategic exploration.}

\primary{We're only going to talk about the first one.  And to help us, we're going to use an example.}

\newpage

\subsection{Scenario 2: John Cage (3 min)}

\musician{PLAY A NOTE.}

\tsdirectionforall{PAUSE 4}

\primary{The example is a person.  Someone from the arts. [PAUSE 1.]  We're going to use John Cage.}

\tstechnical{R: SLIDE: Photo of John Cage.}

\tsdirectionforall{Face the screen with your bodies.}

\primary{[GESTURE AT THE SECONDARY GROUP TO DIRECT ATTENTION TO THEM.] Background.}

\tsdirectionforall{PRIMARY: Walk along the purple tape to the triangle, then face the centre of the hall.}

\secondaryone{John Milton Cage Jr. was an American composer.  He was born in Los Angeles in 1912 and died in New York in 1992.  He studied under composer Arnold Schoenberg [SHERN-berg].  In the early 1940s, he moved to New York, where he met Marcel Duchamp [mar-SELL doo-SHOMP] and other artists.  [TURN ONE QUARTER TO THE LEFT.]}

\secondarytwo{He played chess with Duchamp.  There's a Club connection to Cage and Duchamp and chess. In 1968, our own Donald Gillies ran the production of ``Reunion'' at Ryerson, where Cage and Duchamp played chess and the moves triggered sounds played by unseen musicians.  [TURN ONE-QUARTER TO THE RIGHT.]}

\tstechnical{R: SLIDE: Donald Gillies.}

\secondarythr{Cage married in his early twenties, but divorced his wife in 1945 and spent the rest of his life with Merce Cunningham, a dancer and choreographer.  Later in life, he became a Buddhist.  He followed a macrobiotic diet, and took his own food with him when he travelled.   [TURN ONE-QUARTER TO THE LEFT.]}

\secondaryone{He used chance into his work.  These are ``aleatory'' [AYE-lee-uh-tory] compositions.  He would often roll dice to determine what would happen next.  He used the \textit{I Ching} [EE CHING] as a compositional tool.  [FACE THE CENTRE.]}

\secondarytwo{He was friends with contemporary visual artists like Robert Rauschenberg [ROWSH-en-berg], who once made a work by taking a pencil drawing by Willem de Kooning [WILL-um de KOO-ning] and erasing everything on it.  That's a very Cage thing to do. [FACE THE CENTRE.]}

\secondarythr{He smoked a lot.  He laughed a lot.  He organized the first Happenings.  He was a mushroom expert: a mycologist [my-COLL-o-jist].  He composed the silent piece.  [FACE THE CENTRE.]}

\tsdirectionforall{PAUSE 2}

\primary{The silent piece.  That's \textit{4' 33''} [four minutes and thirty-three seconds].  [WALK ALONG THE PURPLE TAPE AND STOP ON THE X.]  This is the score.  [GESTURE TO THE SCREEN WITH A SWEEP OF YOUR ARM.]}

\tstechnical{R: SLIDE: Show the score.}

\primary{It has three movements.  Each is the word ``Tacet'' [TASS-it].  This tells the musician not to play.  It was first performed by pianist David Tudor in 1952.  He came out on stage and sat at the piano.}

\tsdirectionforall{PRIMARY and all SECONDARY look at TERTIARY on stage.}

\tertiaryone{In the first movement, he didn't play the piano.}

\tertiarytwo{In the second movement, he didn't play the piano.}

\tertiarythr{In the third movement, he didn't play the piano.}

\primary{After four minutes and thirty-three seconds of not playing, he stood up, and the piece was over.  That's the silent piece.}

\secondaryone{Except ... [RAISE ONE FINGER OVER YOUR HEAD]}

\tsdirectionforall{PAUSE 2}

\secondarytwo{... it's not ... [SHRUG YOUR SHOULDERS]}

\tsdirectionforall{PAUSE 3}

\secondarythr{(\textit{firmly}) ... silent.}

\tsdirectionforall{PAUSE 4}

\secondaryone{Even though the pianist wasn't playing ... [LOWER YOUR FINGER]}

\secondarytwo{... the audience still heard sounds ...}

\secondarythr{... from themselves, and the room, and the whole environment.}

\primary{As Cage said, ``There's no such thing as silence.''}

\tsdirectionforall{SECONDARY and TERTIARY: Give an exaggerated shrug and knowing wink to an audience member of your choice.}

\newpage

\subsection{Scenario 3: Authority is Constructed and Contextual (5 min)}

PIANIST:  [PLAY A NOTE.]

[PAUSE 4]

\primary{Back to the concept:  Authority is constructed and contextual.}

\tstechnical{R: SLIDE: Show ``Authority is constructed and contextual''.}

\tstechnical{Use a video?}

\secondaryone{Authority---}

\secondarytwo{---is constructed---}

\secondarythr{---and contextual.}

[PAUSE 1]

\tertiaryone{Authority.}

\tertiarytwo{Constructed.}

\tertiarythr{Contextual.}

[PAUSE 1]

\primary{This is the concept in question.  This is the \textit{frame} we are going to investigate and try to understand.  Let's look at the definition.}

\tstechnical{R: SLIDE: Show ``Information resources reflect'' paragraph.}

\primary{One should never read a block of text from a screen, but they're going to read a block of text from a screen.}

\secondaryone{``Information resources reflect their creators’ expertise and credibility, and are evaluated based on the information need and the context in which the information will be used.''}

\secondarytwo{``Authority is constructed in that various communities may recognize different types of authority.''}

\secondarythr{``It is contextual in that the information need may help to determine the level of authority required.''}

\tstechnical{R: SLIDE: Show ``Information resources reflect'' paragraph with first line bold: ``Information resources reflect their creators’ expertise and credibility, and are evaluated based on the information need and the context in which the information will be used.''}

\primary{First of all, what's an information resource?  Define it, please.}

\tertiaryone{A book.  A web site.  A documentary on TV.  A Facebook post.  An article in an academic journal.  A thread on Twitter.  A newspaper report.  A pamphlet your doctor gave you.  Could be a person!  Like your cousin who knows about really good foreign-language TV shows.}

\primary{How do they reflect their creators' expertise and credibility?}

\tertiarytwo{The book was written by someone who's spent years on the topic, and it's published by a major publisher.  The web site has recipes tested by a cook you like.  The post on Facebook was made by someone who was actually at the event.}

\tertiarythr{The academic article is by a team of people at a research laboratory at a university.   The thread on Twitter is by a cabinet minister, about the government's response to an emergency.}

\primary{And how you use these depends on what you need, when you need it, and how you'll use it.}

\secondaryone{If I'm planning a vacation in a city I've never been to before, I'd go to some web sites, get some travel guides, and ask people I know if they've been there.  That's enough for me as a \textit{tourist}.  But if I was thinking about doing \textit{business} there, that's a whole different thing.  I need information about the economics, demographics, regulations and so on.}

\secondarytwo{Or say I had to write something about the Group of Seven.  If I just needed to check who the ninth and tenth members were, the Wikipedia entry is enough.  But if I'm studying Canadian art history at university, I'd need to read books by art historians and other experts.}

\primary{Experts.  Let's remember that word.  We've got authority and we've got expertise.  But ...}

\tstechnical{R: SLIDE: Show ``Information resources reflect'' paragraph with second line bold: ``Authority is constructed in that various communities may recognize different types of authority.''}

\primary{Authority isn't absolute.  Different people can recognize different types of authority in different situations.  It's socially constructed. }

\secondarythr{People can select their own authorities, based on religion or politics or culture.  Some authorities have that position because of a role that is more or less universally respected, like nurses.  Or maybe it's about power, like the principal in an elementary school.}

\primary{And the last part ...}

\tstechnical{R: SLIDE: Show ``Information resources reflect'' paragraph with third line bold:  ``It is contextual in that the information need may help to determine the level of authority required.''}

\primary{If I want to know what A.J.~Casson was like as a person, someone who knew Cass---a member here who had lunch with him, and maybe visited his house to buy a painting---is one kind of authority.   But if I want to know about the influence the Group has had on Canadian art over the last hundred years, that's different.  I'd look to different authorities.}

\tstechnical{R: SLIDE: Show ``Authority is constructed and contextual''.}

\primary{This frame comes with some \textit{knowledge practices}.  Such as:}

\secondaryone{People who understand this frame will be able to define different types of authority, for example subject expertise, societal position or personal experience.}

\secondarytwo{They know indicators that help determine an authority's credibility.}

\secondarythr{They know that there may be scholars who are widely acknowledged as authorities in an areas, but still, other scholars might challenge them.}

\primary{And they know they can develop their own authoritative voices, and this will develop over time, in relation to other authorities.  Also, there are \textit{dispositions}.  People who understand this frame will maintain an open mind.}

\secondaryone{They will look for authoritative sources, and remember authority doesn't need to come from a university degree.}

\secondarytwo{They will be aware of who is saying someone is an authority, and why.}

\secondarythr{They will be aware of their own attitudes and biases, and think about them.}

[PAUSE 1]

\primary{Now the beginning is in place.  We know about information literacy.}

\secondaryone{Find.}

\tertiaryone{Use.}

\secondarytwo{Understand.}

\tertiarytwo{Evaluate.}

\secondarythr{Integrate.}

\tertiarythr{Share.}

\primary{We know about this frame:  \textit{Authority is constructed and contextual}.  We've looked at the definition, considered some examples, and we have some ideas about how someone who understands this concept would behave when they're looking for information.}

\newpage

\subsection{Scenario 4: Cage and music (3 min)}

PIANIST:  [PLAY A NOTE.]

[PAUSE 4]

\primary{Now we go back to our example:  John Cage.  We're going to look a little bit more at his music, then we're going to talk about mushrooms.  Then we're going to tie it all together.}

\tstechnical{R: SLIDE: Photo of John Cage.}

\primary{We saw \textit{4' 33''}, but he did a lot more than that.}

...

...

\newpage

\subsection{Scenario 5: Cage and mushrooms (6 min)}

PIANIST:  [PLAY A NOTE.]

[PAUSE 4]

\primary{Now we're going to talk about mushrooms.  Mushroom fact number one.  Mushrooms are not plants, they are \textit{fungi}.  }

\secondaryone{So are yeast and mold.  In biology, plants are a kingdom, animals are a kingdom, and fungi are a kingdom.  They don't use photosynthesis.  They get their energy by eating things like wood.}

\primary{Mushroom fact number two.  A mushroom is the fruiting body of a fungus.}

\secondarytwo{It will release spores, which is how the fungus reproduces.  They don't have seeds.}

\primary{Mushroom fact number three.  There is an awful lot going on underground you can't see.}

\secondarythr{Mushrooms grow out of a very complex network structure in the ground called a \textit{mycelium} [my-SEE-lee-um].  It looks like a lot of thin white threads.}

\primary{Most importantly, mushroom fact number zero.  If you don't know that a mushroom is edible, don't eat it!  Mushrooms in the grocery store:  safe.  Mushrooms in the forest:  unknown!  Consult an expert.  Consult an \textit{authority}.}

\tstechnical{R: SLIDE: Photo of John Cage.}

\primary{Someone like John Cage.  }

\newpage

\subsection{Scenario 6: Swiss roll (5 min)}

Tie it all together about Cage and authority and how it is constructed and contextual.

\newpage

\subsection{Scenario 7: 4' 33'' (6 min)}

\tsdirectionforall{Instructions on where to move.}

\tsdirectionforall{Put down your score and pay attention to PRIMARY, who will conduct.}

\primary{[Press play on the device and follow the instructions.  If you cannot hear anything after a few seconds, raise a hand and look for someone in a white lab coat.]}

\tsdirectionforall{When the piece is over, wait until the audience has reacted and had time to cough and move.  Wait until the room is quiet again.}

\newpage

\subsection{Coda (1 min)}

\primary{Thank you.}

\newpage

\subsection{Curtain call}

\tsdirectionforall{What to do for the curtain call.}

\tstechnical{LX: Lights down.  House lights to full.}

\newpage

\section{Bibliography}

List works cited, and perhaps further reading.



\newpage

\section{Credits}

Created by Ashley Williamson and William Denton.

Theatre Science would not exist without the Arts and Letters Club of Toronto.  Thanks to everyone we've ever worked with in any stage production at the club.  We're grateful to the Stage Committee for giving us the time for these performances.  Special thanks to Fiona McKeown and Chris Gardener.  Very special thanks to Michael Spence, who embodies both Theatre and Science.

% Other thanks:  Ramona Baillie, Sophie Bury, Sarah Coysh, Damon Lum, Patti Ryan, Betty Trott.

Theatre Science is CC-BY.  It is licensed under a Creative Commons Attribution 4.0 International license.

\end{document}
