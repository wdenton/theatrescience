\newcommand{\blankspace}{\_\_\_\_\_\_\_\_}

% Performer letter is capped lowercase letter, except for when it's for all performers, when it becomes ALL.
% And if it's st then it becomes 'Ss and Ts'.
% Maybe there's a nicer way to handle this?
\newcommand{\performerletter}[1]{%
  \IfEqCase{#1}{%
    {performers}{ALL}%
    {st}{Ss and Ts}%
    {mst}{M, Ss and Ts}%
    {ps}{P and Ss}%
    {nots1}{ALL but S1}%
    {nots2}{ALL but S2}%
    {tertiary}{Ts}%
}%
    [\MakeUppercase{#1}]%
}

 %  \ifthenelse{\equal{#1}{performers}}{ALL}{\MakeUppercase{#1}}%

  % \ifthenelse{\equal{#1}{performers}}%
  %   {ALL}%
  %   {\ifthenelse{\equal{#1}{st}}%
  %     {Ss and Ts}%
  %       {\ifthenelse{\equal{#1}{ps}}%
  %       {P and Ss}%
  %       {\MakeUppercase{#1}}%
  %   }}%

% \newcommand{\musician}[1]{    \ifthenelse{ \equal{\speaker}{musician}    }{\textbf{M: #1}}{M: #1}}
% \newcommand{\primary}[1]{     \ifthenelse{ \equal{\speaker}{primary}     }{\textbf{P: #1}}{P: #1}}
% \newcommand{\secondaryone}[1]{\ifthenelse{ \equal{\speaker}{secondaryone}}{\textbf{S1: #1}}{S1: #1}}
% \newcommand{\secondarytwo}[1]{\ifthenelse{ \equal{\speaker}{secondarytwo}}{\textbf{S2: #1}}{S2: #1}}
% \newcommand{\secondarythr}[1]{\ifthenelse{ \equal{\speaker}{secondarythr}}{\textbf{S3: #1}}{S3: #1}}
% \newcommand{\tertiaryone}[1]{ \ifthenelse{ \equal{\speaker}{tertiaryone} }{\textbf{T1: #1}}{T1: #1}}
% \newcommand{\tertiarytwo}[1]{ \ifthenelse{ \equal{\speaker}{tertiarytwo} }{\textbf{T1: #1}}{T2: #1}}
% \newcommand{\tertiarythr}[1]{ \ifthenelse{ \equal{\speaker}{tertiarythr} }{\textbf{T1: #1}}{T3: #1}}

\newcommand{\speak}[2]{\ifthenelse{\equal{\speaker}{#1}}{\textbf{\performerletter{#1}: #2}}{\performerletter{#1}: #2}}

% \ifthenelse{\equal{\speaker}{musician}}{%
%   \newcommand{\directionboxweight}{5pt}
% }{%
%   \newcommand{\directionboxweight}{1pt}
% }

% Stage directions are in small caps.
\newcommand{\direction}[1]{[\textsc{\lowercase{#1}}]}

%%
%% Lots of whitespace for performers
%%

% Box formatting, so they have nice lines and are well padded
\setlength{\fboxrule}{2pt}
% More padding for performers.
\ifthenelse{\equal{\speaker}{a}}{\setlength{\fboxsep}{1em}}{\setlength{\fboxsep}{2em}}

% Put a bigger vspace around boxes for performers
% Call with \vspace{\myboxspace}
\newlength{\myboxspace}
\setlength{\myboxspace}{0.4\baselineskip}
\ifthenelse{\equal{\speaker}{a}}{\setlength{\myboxspace}{0.2\baselineskip}}{}

% Use a newpage between sections for performers, but not for the audience (to save space in the booklet score).
\newcommand{\newpageforperformers}{\ifthenelse{\equal{\speaker}{a}}{}{\newpage}}

% Increase the line spacing for performers
\ifthenelse{\equal{\speaker}{a}}{}{\renewcommand{\baselinestretch}{1.2}}

%%
%% Commands
%%

% tsdirectionforall = Direction for All
% \newcommand{\tsdirectionforall}[1]{\setlength{\fboxrule}{2pt}\vspace{0.3\baselineskip}\fbox{\parbox{0.9\textwidth}{\textbf{#1}}}\setlength{\fboxrule}{1pt}\vspace{0.3\baselineskip}}

% \newtoggle{thickbox}
% \ifstrequal{\speaker}{musician}{\toggletrue{thickbox}}{}

% For directions for all secondaries.
\newtoggle{secondary}
\ifthenelse{\equal{\speaker}{s1}}{\toggletrue{secondary}}{}
\ifthenelse{\equal{\speaker}{s2}}{\toggletrue{secondary}}{}
\ifthenelse{\equal{\speaker}{s3}}{\toggletrue{secondary}}{}

% For directions for all tertiaries.
\newtoggle{tertiary}
\ifthenelse{\equal{\speaker}{t1}}{\toggletrue{tertiary}}{}
\ifthenelse{\equal{\speaker}{t2}}{\toggletrue{tertiary}}{}
\ifthenelse{\equal{\speaker}{t3}}{\toggletrue{tertiary}}{}

% For directions for all performers, where a box should be thick
% for everyone *except* the audience, and Q and R.
\newtoggle{performers}
\toggletrue{performers}
\ifthenelse{\equal{\speaker}{a}}{\togglefalse{performers}}{}
\ifthenelse{\equal{\speaker}{q}}{\togglefalse{performers}}{}
\ifthenelse{\equal{\speaker}{r}}{\togglefalse{performers}}{}

% For P and the Ss.
\newtoggle{ps}
\togglefalse{ps}
\ifthenelse{\equal{\speaker}{p}}{\toggletrue{ps}}{}
\ifthenelse{\equal{\speaker}{s1}}{\toggletrue{ps}}{}
\ifthenelse{\equal{\speaker}{s2}}{\toggletrue{ps}}{}
\ifthenelse{\equal{\speaker}{s3}}{\toggletrue{ps}}{}

% For Ss and Ts
\newtoggle{st}
\togglefalse{st}
\ifthenelse{\equal{\speaker}{s1}}{\toggletrue{st}}{}
\ifthenelse{\equal{\speaker}{s2}}{\toggletrue{st}}{}
\ifthenelse{\equal{\speaker}{s3}}{\toggletrue{st}}{}
\ifthenelse{\equal{\speaker}{t1}}{\toggletrue{st}}{}
\ifthenelse{\equal{\speaker}{t2}}{\toggletrue{st}}{}
\ifthenelse{\equal{\speaker}{t3}}{\toggletrue{st}}{}

% For M, Ss and Ts.
\newtoggle{mst}
\togglefalse{mst}
\iftoggle{st}{\toggletrue{mst}}{}
\ifthenelse{\equal{\speaker}{m}}{\toggletrue{mst}}{}

% Everyone except S2 (and the audience)
\newtoggle{nots1}
\toggletrue{nots1}
\ifthenelse{\equal{\speaker}{a}}{\togglefalse{nots1}}{}
\ifthenelse{\equal{\speaker}{s1}}{\togglefalse{nots1}}{}

% Everyone except S2 (and the audience)
\newtoggle{nots2}
\toggletrue{nots2}
\ifthenelse{\equal{\speaker}{a}}{\togglefalse{nots2}}{}
\ifthenelse{\equal{\speaker}{s2}}{\togglefalse{nots2}}{}

\newtoggle{thickbox}

% directionbox
%\newcommand{\directionbox}[1]{\setlength{\fboxrule}{\directionboxweight{}}\vspace{0.5\baselineskip}\fbox{\parbox{\textwidth}{#1}}\setlength{\fboxrule}{1pt}\vspace{0.5\baselineskip}}
\newcommand{\directionbox}[2]{%
  \ifthenelse{\equal{\speaker}{#1}}{\toggletrue{thickbox}}{\togglefalse{thickbox}}

  % If the direction is for secondary, toggle the thickbox on if the secondary toggle is set.
  \ifthenelse{\equal{secondary}{#1}}{%
      \iftoggle{secondary}%
          {\toggletrue{thickbox}}%
          {}%
      }{}

  % If the direction is for performers, toggle the thickbox on.
  \ifthenelse{\equal{performers}{#1}}%
    {\iftoggle{performers}{\toggletrue{thickbox}}{}}%
    {}

  % If the direction is for secondaries and tertiaries, toggle the thickbox on if the st toggle is on.
  \ifthenelse{\equal{st}{#1}}%
    {\iftoggle{st}{\toggletrue{thickbox}}{}}%
    {}

  % Tertiaries
  \ifthenelse{\equal{tertiary}{#1}}%
    {\iftoggle{tertiary}{\toggletrue{thickbox}}{}}%
    {}

  % MST
  \ifthenelse{\equal{mst}{#1}}%
    {\iftoggle{mst}{\toggletrue{thickbox}}{}}%
    {}

  % PS
  \ifthenelse{\equal{ps}{#1}}%
    {\iftoggle{ps}{\toggletrue{thickbox}}{}}%
    {}

  % PS
  \ifthenelse{\equal{nots1}{#1}}%
    {\iftoggle{nots1}{\toggletrue{thickbox}}{}}%
    {}

  % PS
  \ifthenelse{\equal{nots2}{#1}}%
    {\iftoggle{nots2}{\toggletrue{thickbox}}{}}%
    {}

% \toggletrue{thick}
  % \ifthenelse{\equal{musician}{musician}{\toggletrue{thick}}{\togglefalse{thick}}}
  % \setlength{\fboxrule}{\ifthenelse{\equal{#1}{\speaker}}{5pt}{1pt}}%
  %\setlength{\fboxrule}{\ifthenelse{\equal{\speaker}{#1}}{5pt}{1pt}}%
  \setlength{\fboxrule}{\iftoggle{thickbox}{2pt}{1pt}}%
  \vspace{\myboxspace}% \vspace{0.3\baselineskip}%
  \fbox{\parbox{0.9\textwidth}{\iftoggle{thickbox}{\textbf{\performerletter{#1}: #2}}{\performerletter{#1}: #2}}}%
  \setlength{\fboxrule}{1pt}%
  \vspace{\myboxspace}%\vspace{0.3\baselineskip}
}

% tstechnical = Technical direction
% \newcommand{\tstechnical}[1]{\dbox{\vspace{0.3\baselineskip}\parbox{0.9\textwidth}{\textit{#1}}}\vspace{0.3\baselineskip}}

\newcommand{\technicalbox}[2]{%
  \ifthenelse{\equal{\speaker}{#1}}{\toggletrue{thickbox}}{\togglefalse{thickbox}}
  \vspace{0.3\baselineskip}
  \dbox{\parbox{0.9\textwidth}{%
    \iftoggle{thickbox}{\textbf{\textit{\performerletter{#1}: #2}}}{\textit{\performerletter{#1}: #2}}}
  }
  \vspace{0.3\baselineskip}
}

\newcommand{\playanote}{\directionbox{m}{When the others have stopped, play a note.}}

\newcommand{\notewait}{\directionbox{p}{Wait until it is barely possible to hear the piano.}}

% \newcommand{\pronounce}[2]{«\dashuline{#2} #1»} % « »
\newcommand{\pronounce}[2]{#1 {\footnotesize «~#2~»}} % « » \dashuline{#2}

% Bib formatting
%
% This lets me use a {mybiblist} environment to list the bibliographies,
% which are then formatted to look Chicago style.
% Taken from https://tex.stackexchange.com/q/148801/13866
\newenvironment{mybiblist}{%
  % \fontsize{10}{12}\selectfont
  % \footnotesize
  \begin{list}{}{%
      \setlength{\itemindent}{-2em}%
      \setlength{\leftmargin}{2em}%
      \setlength{\itemsep}{0pt}%
      \setlength{\parsep}{\parskip}%
      \setlength{\topsep}{\parskip}%
    }%
    \setlength{\parindent}{-2em}%
  \item[]%
  }{\end{list}}
