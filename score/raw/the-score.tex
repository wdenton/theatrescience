\section{List of performers}

\begin{description}[align=right,labelwidth=2cm]

  \item [A] the audience
  \item [M] one musician
  \item [P] one primary speaker
  \item [S] three secondary roles
  \begin{description}[align=right,labelwidth=2cm]
    \item [S1] can go up the stairs to the stage; does one ``Mushroom Verbatim''
    \item [S2] always on the floor; does one ``Mushroom Verbatim''
    \item [S3] limited movement; stays in the podium zone
  \end{description}
  \item [T] three tertiary roles
  \begin{description}[align=right,labelwidth=2cm]
    \item [T1] limited movement on stage
    \item [T2] limited movement on stage
    \item [T3]  stage right; does not move; can remain seated
  \end{description}
  \item [Q] technical person in the booth
  \item [R] technical person on the floor

\end{description}

P does about 45\% of the speaking, while the secondary group shares another 45\% and the tertiary group shares 10\%.

\newpage

\section{Instructions}

Please read this carefully.  It explains how the formatting in the score works, so you will know what to do and what to say.

\subsection*{Formatting}

Boxes have \textit{stage directions}, which are instructions to be followed but not spoken.

\tsdirectionforall{A bold box has directions for you to follow.}

\vspace{0.5\baselineskip}
\fbox{\parbox{0.9\textwidth}{A regular box has directions for other people.  Do not perform these.}}
\vspace{0.5\baselineskip}

% \setlength{\fboxrule}{1pt}

\tstechnical{A dashed box has technical cues, which are also in italics to set them apart. Only Q and R will act on them.}

Your dialogue may include very brief directions.  These are shown with square brackets and small caps: \direction{Like this}.

Your dialogue is in bold.  Here is an example.

\begin{quote}

L:  The show tonight is in the Great Hall at the Arts and Letters Club.

\textbf{\performerletter{\speaker}:  \direction{Pause 1} We will be making use of the whole space, in and around you, but no audience member will be involved in the show.}

M:  The performance will last about forty minutes.

\end{quote}

\subsection*{Pauses}

\direction{Pause 1} means pause for a count of one, which you might time by saying to yourself, ``One one thousand.''.  \direction{Pause 2} means pause for a count of two, which you might time by saying to yourself, ``One one thousand, two one thousand.''  \direction{Pause 4}, you can imagine, will seem fairly long.

Don't rush a pause.  The audience will wait for you.

\subsection*{Pronunciation}

Words that might be hard to pronounce have a phonetic pronunciation before them in square brackets, with capital letters indicating emphasis:

\begin{quote}

  Composer Arnold \pronounce{Schoenberg}{SHERN-berg}.

  Cage was a mushroom expert: a \pronounce{mycologist}{my-COLL-o-jist}.

\end{quote}


Only say the word once.

If you are unsure of the pronunciation, give it your best shot, but don't worry if you make a mistake.  Try again or just move on.  If the audience noticed, they will forget quickly.

``Fungi'' can be pronounced with a hard or soft \textit{g} and you can say the final \textit{i} so it sounds like ``ee'' or ``aye.''  We like to say it with a hard G so ``fungi'' sounds like ``funky.''

The John Cage composition \textit{4' 33''} should always be pronounced in full as ``four minutes and thirty-three seconds.''

\subsection*{For the musician}

We use the piano to help indicate when a new section begins.  You will see these instructions:

\directionbox{m}{Wait until everyone has stopped moving, then play a note.}

To play a note, do this:  pick any key on the piano, hit it hard, and keep pressing the key down until you can no longer hear any sound from it, even if performers start talking.  When you can't hear the note any more, wait a few more seconds to be sure, then lift your finger up.

Then follow the score and wait for the next scenario.

\newpage

\section{Setup}

\subsection*{Requirements}

\begin{itemize}

  \item A piano or any other instrument that can make a lingering sound.
  \item A projector and a screen, to show a PDF.
  \item Headphones and a way of playing audio on them.

\end{itemize}

\newpage

\subsection*{Technical}

\technicalbox{r}{Setup instructions here.  Projector, laptop, screen, etc.}

\technicalbox{q}{LX: House lights up.  Work lights off.  Lighting board on.  GM \(\uparrow\) 4. Two-scene mode.  Flash mode off.  A \(\uparrow\) 10.  B \(\downarrow\) 10.}

\technicalbox{q}{SX:  Amp on.  Sound board on.  Master audio level to the mark.  Laptop headphone audio to ¾.  Plug laptop into 2 TK in the back.  2 TK input level to 5.}

\technicalbox{q}{SX: Prepare the thirty-minute introductory background sound with time announcements.}

\technicalbox{q}{SX: At 1930, play the thirty-minute introductory background.  First announcement is about ten seconds in.  Adjust volume as needed through the half hour.  It gets louder as it goes, and should be noticeably loud at the finish.  It will end at 2000.}

\newpage

\subsection*{Cast}

\tsdirectionforall{Instructions for the cast: where to stand or sit, etc.}

\newpage

\section{Score}

\subsection{Scenario 0: Announcements} % 3 minutes

\technicalbox{q}{SX: Play the introductory audio.}

\technicalbox{q}{LX: House lights off.}

\technicalbox{q}{SX: Announcement on microphone.  \\
  ``Ladies and gentleman, performing Theatre Science tonight are:\\
  Thomas Gough as Primary, \\
  Michelle Hogan-Walker as Secondary One, \\
  Lorna Kelly as Secondary Two, \\
  John Rammell as Secondary Three, \\
  \blankspace as Tertiary One, \\
  Damon Lum as Tertiary Two, \\
  Lucy Brennan as Tertiary Three, \\
  Rob Prince as the Musician, \\
  William Denton as Q and Ashley Williamson as R. \\
  You may follow along with the score provided.  \\
  Theatre Science has begun.''}

% Use \blankspace for blanks.

\technicalbox{q}{Hit the gong loudly.}

\newpage

\subsection{Scenario 1: Introduction} % 4 minutes

\technicalbox{q}{LX: House lights up to 65\%.}

\directionbox{performers}{Do these movements simultaneously.}

\directionbox{m}{Follow the red line until you are behind the piano, then sit on the piano stool.}

\directionbox{p}{Follow the yellow line.  Stop on the square.}

\directionbox{s1}{Follow the blue line on the right and stop on the X.}

\directionbox{s2}{Follow the green line on the left and stop on the X, then face P.}

\directionbox{s3}{Follow the pink tape and stop on the chevrons, then face P.}

\directionbox{t1}{Follow the orange line until you come to your marked position.  Stand there.}

\directionbox{t2}{Follow the orange line until you come to your marked position.  Stand there.}

\directionbox{t3}{Follow the orange line until you come to your marked position.  Sit there.}

\playanote{}

\directionbox{p}{When you can no longer hear the note, wait one more second.  Slowly raise your arms over your head, then slowly lower them back to your sides.}

\directionbox{performers}{Copy P's arm movements as best you can.}

\speak{p}{This show is about a concept from the field of library and information science.  The concept is:  \textit{Authority is constructed and contextual.} }

\technicalbox{r}{SLIDE: ``Authority is constructed and contextual.''}

\speak{p}{This comes from a field of librarianship called \textit{information literacy}.}

\speak{s1}{\textit{Literacy}, on its own, can be defined in a very simple way as being able to read and write.  But the world is a lot more complicated now than it was five thousand years ago.  There is writing all around us, on paper or on screens.}

\speak{s2}{There are many different kinds of literacies.  The equivalent with numbers is \textit{numeracy}.  Being able to read and follow maps is a kind of literacy. There is also \textit{digital literacy} and \textit{media literacy}. }

\speak{s3}{For librarians the key concept is \textit{information literacy}.  Here's a definition.  It's dense, but we'll break it down.}

\technicalbox{r}{SLIDE: ``Information literacy is the set of \ldots''}

\directionbox{performers}{Note that some people move while P speaks.}

\directionbox{s1}{Follow the blue line and stop on the blue triangle.}

\directionbox{s2}{Follow the green line and stop on the green triangle.}

\directionbox{s3}{Follow the pink line to the blue line, then stop on the star.}

\directionbox{p}{Turn to face the screen, then read while the others move.}

\speak{p}{``Information literacy is the set of integrated abilities encompassing the reflective discovery of information, the understanding of how information is produced and valued, and the use of information in creating new knowledge and participating ethically in communities of learning.''  \direction{Pause 1}  Citation.}

\speak{t1}{Association of College and Research Libraries.}

\speak{t2}{\textit{Framework for Information Literacy for Higher Education}.}

\speak{t3}{2015.}

\speak{p}{In other words, information literacy is the ability to do these with information:}

\speak{t1}{Find.}

\speak{t2}{Use.}

\speak{t3}{Understand.}

\speak{t1}{Evaluate.}

\speak{t2}{Integrate.}

\speak{t3}{Share.}

\speak{p}{They don't have to all be done at the same time, or in that order.}

\speak{s1}{Just reading the news requires you to understand, evaluate and integrate, but the news is coming at you all the time, you don't have to go out and find it.}

\speak{s2}{And it happens in different contexts.  Helping a kid with a homework assignment about trees is a lot easier than understanding changes to provincial policy about the Green Belt, but the steps are the same.}

\speak{s3}{A lot depends on people.  Information resources don't have to be written down.  If they are written, they're written by people.  They could be on film or audio.  But sometimes they actually \textit{are} people.  We turn to others for information or advice or expertise.}

\directionbox{p}{Move along the yellow line to the X, then turn to face the screen.}

\speak{p}{Librarians used to think of information literacy as being very much about using computers to do research.  And they had long lists of specific tasks that people should be able to do.  But as they taught information literacy to students in colleges and universities, and as the world changed, they found this approach wasn't the best for:}

\speak{t1}{Find.}

\speak{t2}{Use.}

\speak{t3}{Understand.}

\speak{t1}{Evaluate.}

\speak{t2}{Integrate.}

\speak{t3}{Share.}

\speak{p}{So in 2016 a group wrote \textit{Framework for Information Literacy for Higher Education}.  It has six ``frames.''  These frames are ``threshold concepts.''  That's the idea that there is a change that happens when you go from a beginner's understanding of something to seeing the big picture of how it all fits together, with all the nuances and subtleties. The six frames are these.}

\speak{s1}{Authority is constructed and contextual.}

\speak{s2}{Information creation as a process.}

\speak{s3}{Information has value.}

\speak{s1}{Research as inquiry.}

\speak{s2}{Scholarship as conversation.}

\speak{s3}{Searching as strategic exploration.}

\speak{p}{We're only going to talk about the first one.}

\speak{s1}{Authority---}

\speak{s2}{---is constructed---}

\speak{s3}{---and contextual.}

\speak{p}{\direction{Pause 1} And to help us, we're going to use an example.}

\directionbox{p}{Walk back along yellow tape, cross to the green square, then stop and face centre.}

\directionbox{s1}{Follow the blue line and stop on the chevrons.  Face the stage.}

\directionbox{s2}{Follow the green line to the X.  Cross to the pink X.}

\directionbox{s3}{Follow the blue line and stop on the blue square.  Face the centre.]}

\newpage

\subsection{Scenario 2: John Cage} % 3 minutes

\playanote{}

\notewait{}

\speak{p}{To help us, we're going to use an example.  An artist.  A musician.  \direction{Pause 1}  John Cage.}

\technicalbox{r}{SLIDE: Photo of John Cage.}

\directionbox{performers}{Everyone except P face the screen with your bodies.  P does not move.}

\speak{p}{\textit{Grove Music Online} is the largest and most respected reference source for music.  Here is what it says.}

\speak{t1}{``Born Los Angeles, September 5, 1912; died New York, August 12, 1992.  American composer. One of the leading figures of the postwar avant garde.''}

\speak{t2}{``The influence of his compositions, writings and personality has been felt by a wide range of composers around the world.''}

\speak{t3}{``He had a greater impact on music in the 20th century than any other American composer.''}

\directionbox{p}{Gesture at the secondary group to direct attention to them.}

\speak{p}{More, please.}

\speak{s1}{In his twenties he studied for two years under composer Arnold \pronounce{Schoenberg}{SHERN-berg}, who later said that Cage was ``not a composer, but an inventor---of genius.''  Early success led him to Chicago, where he met artist Max Ernst.  Ernst invited him to come to New York and said they could stay with him and his wife.  (That wife was wealthy art collector Peggy Guggenheim.)}

\directionbox{s1}{Turn one quarter to the right.}

\technicalbox{r}{SLIDE: Donald Gillies.}

\speak{s2}{In New York Cage met Marcel \pronounce{Duchamp}{doo-SHOMP} and often played chess with him.  In fact, there's a Club connection to Cage and Duchamp and chess. In 1968, our own Donald Gillies ran the production of ``Reunion'' at Ryerson, where Cage and Duchamp played chess and the moves triggered sounds played by unseen musicians.  Our own Carol Anderson was in the audience.}

\technicalbox{r}{SLIDE: Photo of John Cage.}

\directionbox{s2}{Turn one-quarter to the right.}

\speak{s3}{He used chance to take himself out of the process of composing.  He would often roll dice to determine what would happen next, and he used the \pronounce{\textit{I Ching}}{ee CHING} as a tool.  He wrote music for percussion, piano, tape, voice, small ensembles, and other instruments.  One piece is written for twelve radios, each played by two people: one controls the tuning and the other the volume and timbre.  Most famously, he used silence.}

\directionbox{s3}{Turn one-quarter to the left.}

\speak{s1}{He was enormously talented and always curious.  He worked very hard.  He was kind man.  Younger composers appreciated the interest and kindness he showed even when they were doing work very different from his.}

\directionbox{s1}{Face the centre.}

\speak{s2}{He appeared in a surprising number of television game shows.  He was a Zen Buddhist.  He laughed a lot.  He organized the first Happenings.  He collaborated with Robert \pronounce{Rauschenberg}{ROWSH-en-burg}, Yoko Ono, and many, many others.  He was a mushroom expert: a \pronounce{mycologist}{my-COLL-o-jist}.}

\directionbox{s2}{Face the centre.}

\speak{s3}{Cage was married to a woman artist for a decade, but spent most of his life with \pronounce{Merce}{MURSE} Cunningham, a dancer and choreographer.  Cage composed many works for Merce's dance company, and together they toured the world.}

\directionbox{s3}{Face the centre.}

\speak{t1}{Chance.}

\speak{t2}{Zen.}

\speak{t3}{Silence.}

\speak{t1}{Laughter.}

\speak{t2}{Mushrooms.}

\speak{t3}{Merce.}

\speak{p}{\direction{Pause 1} Cage is best known for his silent piece.  That's \textit{4' 33''}.}

\directionbox{p}{Cross to the yellow tape and stop on the X.}

\speak{p}{This is the score.}

\directionbox{p}{Gesture overhead.}

\technicalbox{r}{SLIDE: Score of \textit{4' 33''}.}

\speak{p}{It has three movements.  Each is just the word \pronounce{``tacet''}{TASS-it}.  This tells the musician not to play.  It was first performed by pianist David Tudor in 1952, in a concert space in the woods that was open to the outside.  He came on stage and sat at the piano.}

\tsdirectionforall{PRIMARY and all SECONDARY look at TERTIARY on stage.}

\speak{t1}{In the first movement, he didn't play the piano.}

\speak{t2}{In the second movement, he didn't play the piano.}

\speak{t3}{In the third movement, he didn't play the piano.}

\speak{p}{After four minutes and thirty-three seconds of not playing, he stood up, and the piece was over.  That's the silent piece.}

\speak{s1}{Except \ldots}

\directionbox{s1}{Raise one finger over your head.}

\speak{s2}{\direction{Pause 2} \ldots it's not \ldots}

\directionbox{s2}{Shrug your shoulders.}

\speak{s3}{\direction{Pause 3} \ldots silent.}

\speak{s1}{\direction{Pause 4} Even though the pianist wasn't playing \ldots}

\directionbox{s1}{Lower your finger.}

\speak{s2}{\ldots the audience still heard sounds \ldots}

\speak{s3}{\ldots from themselves, and the room, and the whole environment.}

\speak{p}{There's a famous quote from Cage.  Citation first.}

\speak{t1}{Richard Kostelanetz.}

\speak{t2}{\textit{Conversing with Cage}.}

\speak{t3}{2003.}

\speak{p}{The quote is \direction{Pause 1}: ``There's no such thing as silence.''}

\tsdirectionforall{SECONDARY and TERTIARY: Give an exaggerated shrug and knowing wink to an audience member of your choice.}

\directionbox{s1}{Walk the blue line to the pink line and stop on the X.}

\directionbox{s2}{Walk the pink line to the blue line and stop on the X.}

\directionbox{s3}{Walk the blue line towards the screen, then cross to the yellow line and walk to the X.}

\directionbox{p}{Cross to the pink line, then walk on the blue line towards the screen, and stop on the chevrons.}

\newpage

\subsection{Scenario 3: Authority} % 5 minutes

\playanote{}

\notewait{}

\speak{p}{Back to the concept.}

\technicalbox{r}{SLIDE: ``Authority is constructed and contextual''.}

\speak{s1}{Authority---}

\speak{s2}{---is constructed---}

\speak{s3}{---and contextual.}

\speak{t1}{\direction{Pause 1} Authority.}

\speak{t2}{Constructed.}

\speak{t3}{Contextual.}

\speak{p}{\direction{Pause 1} This is the concept in question.  This is the \textit{frame} we are going to try to understand.  Let's look at the definition.}

\directionbox{performers}{Turn to look at the screen.}

\technicalbox{r}{SLIDE: ``Information resources reflect'' paragraph.}

\speak{p}{Please read me that big block of text on the screen.}

\speak{t1}{``Information resources reflect their creators’ expertise and credibility, and are evaluated based on the information need and the context in which the information will be used.''}

\speak{t2}{``Authority is constructed in that various communities may recognize different types of authority.''}

\speak{t3}{``It is contextual in that the information need may help to determine the level of authority required.''}

\speak{p}{Citation.}

\speak{t1}{Association of College and Research Libraries.}

\speak{t2}{\textit{Framework for Information Literacy for Higher Education}.}

\speak{t3}{2015.}

\directionbox{p}{Face the piano.}

\directionbox{s1}{Face the piano.}

\directionbox{s2}{Face the piano.}

\directionbox{s3}{Face the podium.}

\technicalbox{r}{SLIDE: ``Information resources reflect'' paragraph with first line bold.}

\speak{p}{First of all, what's an information resource?  Define it, please.}

\speak{s1}{A book.  A web site.  A documentary.  A Facebook post.  An article in an academic journal.  A thread on Twitter.  A newspaper report.  A pamphlet your doctor gave you.  Could be a person!}

\speak{p}{How do they reflect their creators' expertise and credibility?}

\speak{s2}{The book was written by someone who's spent years on the topic, and it's published by a major publisher.  The web site has recipes tested by a cook you like.  The post on Facebook was made by someone who was actually at the event.}

\speak{s3}{The academic article is by a team of people at a research laboratory at a university.   The thread on Twitter is by a cabinet minister, about the government's response to an emergency.}

\speak{p}{And how you use these depends on what you need, when you need it, and how you'll use it.}

\speak{s1}{If I'm planning a vacation in a city I've never been to before, I'd go to some web sites, get some travel guides, and ask people I know if they've been there.  That's enough for me as a \textit{tourist}.  But if I was thinking about doing \textit{business} there, that's a whole different thing.  I need information about the economics, demographics, regulations and so on.}

\speak{s2}{Or say I had to write something about the Group of Seven.  If I want to check who the ninth and tenth members were, the Wikipedia entry is enough.  But if I'm studying Canadian art history at university, I need books by art historians and other experts.}

\speak{p}{Next part.}

\directionbox{performers}{Turn to look at the screen.}

\technicalbox{r}{SLIDE: ``Information resources reflect'' paragraph with second line bold: ``Authority is constructed \ldots''}

\speak{p}{Authority isn't absolute.  Different people can recognize different types of authority in different situations.  It's socially constructed. }

\speak{s3}{People can select their own authorities, based on religion or politics or culture.  Some authorities have that position because of a role that is more or less universally respected, like nurses.  Or maybe it's about power, like the principal in an elementary school.}

\speak{p}{And the last part.}

\technicalbox{r}{SLIDE: ``Information resources reflect'' paragraph with third line bold:  ``It is contextual in that \ldots''}

\speak{p}{If I want to know what A.J.~Casson was like as a person, someone who knew Cass---maybe a member here who used to have lunch with him, and visited his house to buy a painting---is one kind of authority.   But if I want to know about the influence the Group has had on Canadian art over the last century, that's a different context.  I'd want to talk to art historians.  Though that person who knew Cass could \textit{be} an art historian, and be an authority in two different ways.}

\technicalbox{r}{SLIDE: ``Authority is constructed and contextual''.}

\speak{p}{Two last things about this frame.  There are some things people who are getting better with this concept will do, such as:}

\speak{s1}{They will be able to define different types of authority, for example: subject expertise, societal position or personal experience.}

\speak{s2}{They know indicators that help determine an authority's credibility.}

\speak{s3}{They know that there may be scholars who are widely acknowledged as authorities in an area, but still be challenged by other scholars.}

\speak{p}{And they know they can develop their own authority, and that doing so comes with responsibilities.  Also, there are ways that people will think and act.  People who understand this frame will maintain an open mind.}

\speak{s1}{They will look for authoritative sources, and remember authority doesn't need to come from a university degree.}

\speak{s2}{They will be aware of who is saying someone is an authority, and why.}

\speak{s3}{They will be aware of their own attitudes and biases.}

\speak{p}{\direction{Pause 1} Now the beginning is in place.  We know about information literacy.}

\speak{t1}{Find.}

\speak{t2}{Use.}

\speak{t3}{Understand.}

\speak{t1}{Evaluate.}

\speak{t2}{Integrate.}

\speak{t3}{Share.}

\speak{p}{And we know about the frame.}

\speak{s1}{Authority---}

\speak{s2}{---is constructed---}

\speak{s3}{---and contextual.}

\speak{p}{Now we go back to our example.}

\directionbox{p}{Follow the blue line to the podium, then stand behind it, facing the centre.}

\directionbox{s1}{Walk the pink line to the pink chevrons.}

\directionbox{s2}{Walk the blue line to the green line, then stop on the triangle.}

\directionbox{s3}{Cross to the pink line and stop on the X.}

\newpage

\subsection{Scenario 4: Music} % 3 minutes

\playanote{}

\notewait{}

\speak{p}{John Cage.}

\technicalbox{r}{SLIDE: Photo of John Cage.}

\speak{t1}{Chance.}

\speak{t2}{Zen.}

\speak{t3}{Silence.}

\speak{t1}{Laughter.}

\speak{t2}{Mushrooms.}

\speak{t3}{Merce.}

\speak{p}{We start with his music.  We saw the score of \textit{4' 33''}.  If someone knows one thing about Cage, they know he composed a work of music where there is no music.  That is a very challenging idea.  The piece has been attacked many times over the years.  Even his mother was unsure.  After the premier, she said to composer Earle Brown, ``Now Earle, don't you think that John has gone too far this time?''}

% \speak{s1}{He arrived at it through years of work.  There was a lot of thought and intent behind it.  Even though his music could sound wild and difficult, it was made with care and deliberation.  He knew what he was doing.  And through his life he kept changing what he was doing.}

% \speak{s2}{Cage began to study and compose music in the early 1930s.  Early on he studied under the composer Arnold Schoenberg, who was impressed with Cage's work.  However, harmony was important to Schoenberg but not to Cage, and he soon moved on.  One of his early percussion pieces, ``Third Construction,'' is still commonly performed by ensembles today.}

% \speak{s3}{He invented the prepared piano.  By putting screws and bolts and weather stripping between the strings of a piano, when the keys are hit the sound produced is very different from a normal piano.  Cage said it was ``a percussion ensemble under the control of a single player.''}

% \speak{p}{This all sounds rather long, and will be broken up differently.  There will be one or two quotes from Cage about his approach to sound and music.}

% hard work
% good luck
% enormous talent

% His work was appreciated and he was becoming known.  In Chicago he met the surrealist painter Max Ernst, who invited him to come to New York and stay with him and his wife, Peggy Guggenheim.   Cage and his wife moved there and he had more success.  In 1942 he began to work with choreographer Merce Cunningham, and in 1943 he left his wife for him.  He would compose music for Merce's dances for the rest of his life.

% It was in New York that Cage met Marcel Duchamp and other artists.  He would

% Marcel Duchamp.  Joseph Albers.  Buckminster Fuller.  Joseph Campbell.
% Willem de Kooning.  Robert Rauschenberg.
% Virgil Thomson.  Pierre Boulez.  Morton Feldman.
% Cage introduced Pierre Boulez to Aaron Copeland (in Paris)

% I Ching.

% 4' 33''

% Radio piece.

% John and Yoko.

% Brian Eno.

% Nono p. 160 Conversing. Critivcal of Cage's poloitics

\speak{s1}{I will say something about all the people he worked with, the Happenings, people he influenced.  Major names.}

\speak{s2}{He invented the prepared piano.  I will mention the books he wrote.}

\speak{s3}{I will mention awards, CDs, the John Cage Trust.  General impact now.  Quote from Gavin Bryars.}

\speak{p}{You may never have heard anything composed by John Cage.  It's rare to hear his music without making an effort to go find it.  But I quote \textit{Grove Music Online} again: ``He had a greater impact on music in the 20th century than any other American composer.''  \direction{Pause 1} Next: mushrooms.}

\technicalbox{r}{SLIDE:  Mushrooms.}

\directionbox{s1}{Follow the pink line to the X.}

\directionbox{s2}{Follow the green line towards the stage.  Stop on the X.}

\directionbox{s3}{Follow the pink line towards the entrance curtains, then cross to the yellow square.  Face the centre.}

\speak{p}{Follow the blue line towards the entrance curtains.  Stop on the blue chevrons.}

\newpage

\subsection{Scenario 5: Mushrooms} % 5 minutes

\playanote{}

\notewait{}

\speak{p}{John Cage became interested in mushrooms in 1954, when he was staying at a farmhouse.  He said:}

\speak{s1}{``I found myself living in small quarters with four other people, and I was not used to such lack of privacy, so I took to walking in the woods.  And since it was August, the fungi are the flora of the forest at that time.''}

\speak{s2}{``I was very involved with chance operations in music, and I thought it would just be a very good thing if I could get involved with something where I could not take chances.''}

\speak{p}{Asked if he had any favourite mushrooms, he said:}

\speak{s3}{``I like the ones I have.  If you like the ones you don't have, then you're not happy.''}

\speak{p}{Citation.}

\speak{t1}{Richard Kostelanetz.}

\speak{t2}{\textit{Conversing with Cage}.}

\speak{t3}{2003.}
% Kostelanetz, p. 16.

\speak{p}{Cage quickly became an expert, so much so that four years later he won a lot of money on Italian TV.  While in Milan, on a concert tour, he went on a game show called \textit{Leave or Double}.  Contestants won money by answering questions on a specialty subject.  His was mushrooms.  He appeared several times, building up his winnings.  Finally he had one-and-a-half million \textit{lire}, and in his last appearance he took the risk of answering three more questions to either win five million \textit{lire} (about eight thousand dollars) or lose it all.}

% https://johncagetrust.blogspot.com/2011/04/lascia-o-raddoppia-milan-1959.html
% CITE: https://johncage.org/blog/transcription.html
% CITE: BEGIN AGAIN, p. 168.

\speak{s1}{For the first question, he was shown photographs of mushrooms and had to identify them.  He got each one correct.}

\speak{s2}{Next he was shown a photo of a particular mushroom and had to give its scientific name, the colour and shape of its spores, and the length and width of its spores.  He gave all the right answers.}

\speak{s3}{Finally he was asked to list the twenty-four types of white-spore mushrooms given in a standard text.  Cage said, ``I can enumerate the list alphabetically,'' and then did so.}

\speak{p}{Cage later said, ``That was the first consequential amount of money I'd ever earned.  It was two years later, in 1960, that I began to make some money with my music.''}

\directionbox{s1}{Walk straight to the the orange X in the centre.}

\directionbox{s2}{Walk straight to the orange square in the centre.}

\directionbox{s3}{Walk to the steps.  Sit on the green X facing the audience.}

\directionbox{p}{Move to the yellow star.}

\speak{p}{Mushrooms became a regular part of Cage's music and writing.  We are going to hear two stories from his 1959 work \textit{Indeterminacy}.}

\speak{s3}{My two colleagues are now going to perform ``Mushroom Verbatim.''  One after the other, they are going to put on those headphones and repeat what they hear.  It is not easy to repeat what you are hearing, without preparation.  We will forgive them any mistakes.  First, story number sixty-six.}

\directionbox{s1}{Cross to the plinth and stand in front of the chair.  Here is what is going to happen:  you are going to put on the headphones and then repeat what you hear.  To help people hear your voice, you should look at the top of the stage so your head is raised.  \\
  \\
  Now it is time to get ready.  Put your binder down on the chair, face up so that you can see these words.  Put on the headphones.   Face the stage.  When you are ready, make a ``thumbs'' up sign with each hand.  You will hear two bongs, then the words for you to repeat.}

\tsdirectionforall{While the Mushroom Verbatim is happening, look at the speaker.}

\tstechnical{Q: SX: Play ``Indeterminacy 66'' audio file over headphones.}

\directionbox{s1}{When the story is over, put the headphones back on the plinth, pick up your binder, and return to where you were standing.}

\speak{p}{Next, story number sixty.}

\directionbox{s2}{Cross to the plinth and stand in front of the chair.  Here is what is going to happen:  you are going to put on the headphones and then repeat what you hear.  To help people hear your voice, you should look at the top of the stage so your head is raised.  Now it is time to get ready.  Put your binder down on the chair, face up so that you can see these words.  Put on the headphones.   Face the stage.  When you are ready, make a ``thumbs up'' sign with each hand.  You will hear two bongs, then the words for you to repeat.}

\tsdirectionforall{While the Mushroom Verbatim is happening, look at the speaker.}

\tstechnical{Q: SX: Play ``Indeterminacy 60'' audio file over headphones.}

\directionbox{s2}{When the story is over, put the headphones back on the plinth, pick up your binder, and return to where you were standing, cross to the green tape and go towards the entrance curtains.  Turn onto the blue line and go to the blue square.  Face the piano.}

\directionbox{p}{When S2 is done and starts to move, walk on the yellow tape and stand on the X.  Face the piano.}

\directionbox{s1}{Follow the pink line to the blue line and stand on the chevrons.  Face the piano.}

\directionbox{s3}{Come down the steps.  Follow the blue line to the X.  Face the piano.}

\newpage

\subsection{Scenario 6: Cage and authority} % 5 minutes

\playanote{}

\notewait{}

\speak{p}{Cage was an expert on mushrooms, but he wasn't \textit{just} an expert.  He was also an \textit{authority}, which is different.  Someone can be an expert without being an \textit{authority}.}

\speak{t1}{Authority.}

\speak{t2}{Constructed.}

\speak{t3}{Contextual.}

\speak{s1}{Cage began teaching mushroom identification at the New School in New York.  This led to him helping to found the New York \pronounce{Mycological}{my-co-LOJ-ical} Society, which is still operating today.}

\speak{s2}{He had ``perhaps the most extensive private library ever compiled on the subject'' of mushrooms.  In 1960 he sold wild mushrooms to the Four Seasons restaurant in New York.}

\speak{s3}{However, he never ate magic mushrooms.  He had no interest that kind of experience.}

\speak{p}{Cage was an \textit{authority} on mushrooms.  Definition again, please.}

\tstechnical{R: SLIDE: Show ``Information resources reflect'' paragraph.}

\speak{t1}{``Information resources reflect their creators’ expertise and credibility, and are evaluated based on the information need and the context in which the information will be used.''}

\speak{s1}{We know about his expertise and his credibility.  As for information needs, if you wanted to know about mushrooms, perhaps while out in the forest, he would be a great person to ask.}

\speak{t2}{``Authority is constructed in that various communities may recognize different types of authority.''}

\speak{s2}{He was recognized by other mushroom lovers and researchers for his knowledge and his association work.  He even won awards.}

\speak{t3}{``It is contextual in that the information need may help to determine the level of authority required.''}

\speak{s3}{If you needed a genetic analysis of a fungus, you'd go to a researcher in a lab.  But if you needed to identify a mushroom you'd found in the forest, he could do it.  He could tell you if it was poisonous or not.  You could trust John Cage to stop you from dying from mushroom poisoning.}

\speak{p}{Cage was an authority on mushrooms.  What about music?  Here's some evidence from conceptual artist and musician Yoko Ono, who met Cage in New York in the early sixties.  She displays the independence shown in her art.  She said:  ``I did not immediately take to someone who was considered an authority of some kind.  I mean John Cage was already an authority in some ways, you know; people respected him.  I was just being a bit cynical.''}

% \speak{p}{John Rockwell.}

% \speak{s1}{``Under Mr. Cage's benign influence, younger composers felt free to be free, to dispense with presuppositions about what music is and music be and to explore where their instincts and interests and modern technology took them.''}

% \speak{s2}{``The result has been a lot of trivial trash, of course, but there have also been some masterpiece---usually sounding totally unlike Mr.~Cage's own music, healthily enough---and even whole movements (Minimalism, most prominently) that owe their very existence to Mr.~Cage's liberating example.''}

% \speak{s3}{``Curiously, however, that example has been conveyed primarily through Mr.~Cage's writings (above all the seminal collection \textit{Silence} of 1961 \ldots) and the general example of his personality.''}

% \speak{p}{Citation.}

% \speak{t1}{John Rockwell.  Or something expressing the same idea.}

% \speak{t2}{``The Impact and Influence of John Cage'', in \textit{The New York Times}.}

% \speak{t3}{1987.}

% \speak{p}{That article was written when a new series of recordings of Cage's music was being released.  But it quotes Cage: ``I myself do not use records.  I just listen to the sounds around wherever I happen to be.'' \direction{Pause 2}  As noted, Cage's life and writings are more influential that his musical methods.  Here is a quote from the autobiography of composer Philip Glass.}

\speak{p}{Cage's influence on younger musicians and artists wasn't just in the way he composed.  It also came from his writings, such as his 1961 book \textit{Silence}.  Composer Philip Glass read it.  He said:}

\speak{s1}{``The book \textit{Silence} was in my hands not long after it came out, and I would spend time with [two friends] talking and thinking about it.  As it turned out, it became a way that we could look at what Jasper Johns, Robert Rauschenberg, Richard Serra or almost anybody from our generation or the generation before us did, and we could understand it in terms of how the work existed in the world.''}

\speak{s2}{``The accepted idea when I was growing up was that the late Beethoven quartets or \textit{The Art of the Fugue} or any of the great masterpieces had a platonic identity---that they had an actual, independent existence.  What Cage was saying is that there is no such thing as an independent existence.''}

\speak{s3}{``The music exists between you---the listener---and the object you're listening to.  The transaction of it coming into being happens through the effort you make in the presence of that work.  The cognitive activity is the content of the work.''}

\speak{p}{Citation.}

\speak{t1}{Philip Glass.}

\speak{t2}{\textit{Words Without Music}.}

\speak{t3}{2015.}

\speak{p}{Opposing ideas.  People who have no time for Cage at all.}

\speak{s1}{I will say something.}

\speak{s2}{I will say something.}

\speak{s3}{And I will say something, probably quoting Cage.}

\speak{p}{Cage didn't do his work in isolation.  He had his influences and his collaborators.  He was part of a scene.  But he also helped \textit{create} that scene, and future scenes:  minimalism, conceptual art, Fluxus, the Happenings.  Nuit Blanche owes something to Cage.  The influence is all around.  But at the same time, many people have absolutely no use for him.  He had no effect on them.  Cage is not an authority for them about music, art, life or anything.  Composer and critic Kyle Gann said:}

\speak{s1}{``Yet how to explain the fact that, for hundreds of composers, Cage remains one of the central musical figures of the mid-twentieth century, if not \textit{the} central one? ''}

\speak{s2}{``For thousands trained in American conservatories, Cage is merely a historical irritant, a charlatan against whom their teachers warned them.  For another type, Cage is a model, a basic paradigm for a composer's life.  In America, this is probably the most reliable distinction between what is called the Uptown composer (Uptown as in Manhattan) and the Downtown composer.''}

\speak{s3}{``Uptown composers mention Cage with anger or contempt or condescension; Downtown composers talk of him with reverence, respect, and seriousness.  Perhaps no other composer in history elicits such extremes of approval and disapproval.''}

\speak{p}{Citation.}

\speak{t1}{Kyle Gann.}

\speak{t2}{``No Escape from Heaven: John Cage as Father Figure,'' in \textit{The Cambridge Companion to John Cage}.}

\speak{t3}{2002.}

\speak{p}{Where do you fit in?  Maybe Cage \textit{is} an authority for you: you look to him for lessons about how to find and create music and art and life.  Or maybe he \textit{isn't} an authority for you: nothing of his work is an influence, and he and his followers are completely removed from your practice.}

\speak{s1}{In which case, you have different authorities.  Either way, the authority is constructed.  Different communities recognize different types of authority.  Cage is an authority at Soundscapes and the Music Gallery.  At Opera Atelier and Tafelmusik, he is not.}

\speak{s2}{And it's contextual.  Do you want to let ambient environmental sounds to be part of a performance?  That will lead you right to Cage.  If you want to perform a crowd-pleasing Mozart symphony, you will look elsewhere.}

\speak{s3}{You can make up your own mind about Cage's artistic expertise and whether it is useful for what you need.}

% Doesn't have to be about art and music.
% Perform any disciplined action.
% Quote about attention?

\speak{p}{What if you knew very little about John Cage before tonight?  Maybe you're unsure what to make of all this.  Whichever role you're in, we're going to give you some time to think about it.  And experience it.  \direction{Pause 1}  We're going to end the show with a performance of \textit{4' 33''}.}

\speak{s1}{Walk directly towards the stage and sit on the chair (marked with an X) that is to your left (stage right).}

\speak{s2}{Walk directly towards the stage and sit on the chair (marked with an X) that is to your right (stage left).}

\speak{s2}{Walk directly towards the stage and sit on the steps.}

\tsdirectionforall{Tertiaries: Sit on your chair.}

\directionbox{p}{Stand in front of the plinth (WHERE YOU SHOULD ALREADY BE; CONFIRM).}

\newpage

\subsection{Scenario 7: \textit{4' 33''}} % 6 minutes

\playanote{}

\notewait{}

\speak{p}{I will conduct.}

\speak{s1}{The performers will be three of us, and the three of them on stage, and the musician at the piano.  None of us have rehearsed this.}

\speak{s2}{The conductor will put on headphones, but this is \textit{not} a Mushroom Verbatim.  The conductor will listen to, but \textit{not} repeat, directions on how to conduct us.}

\speak{s3}{We are going to read the instructions in our scores, and then follow them.  As you know, there are three movements.}

\speak{t1}{In the first movement, we won't play an instrument.}

\speak{t2}{In the second movement, we won't play an instrument.}

\speak{t3}{In the third movement, we won't play an instrument.}

\speak{p}{There will be brief pauses between the movements.  We will now prepare.}

\tsdirectionforall{This is what will happen:  You are going to pay close attention to P, who will conduct.  Do not do anything.  Just pay attention and listen.  When the piece is over, P will say, ``That was \textit{4' 33''}.''  Then you will stand and bow with the conductor.  \\
  \\
  Now it is time to begin.  When you are ready, place your score face up on the floor in front of you (the musician can leave the score where it is), sit up straight, and pay attention.}

\directionbox{p}{This is what will happen: You are going to put on the headphones and follow the instructions you hear about how to conduct this piece.  This is \textit{not} a Mushroom Verbatim.  Do not repeat what you hear.\\
  \\
Now it is time to begin.  Stand facing the stage.  Place your score face up on the chair.  Put on the headphones.  When the others in front of you have put their scores on the floor, they are ready.  When you are ready, make a ``thumbs up'' with each hand.  You will hear two bongs, then the instructions.}

\technicalbox{q}{SX: Play audio file 433-instructions.mp3.}

\tsdirectionforall{After P has said ``That was \textit{4' 33''},'' and you have stood and bowed with P, pick up your scores and follow the next directions.}

\directionbox{s1}{Go to the pink chevrons, then follow the blue line to the star.  Face the centre.}

\directionbox{s2}{Follow the blue line to the triangle.}

\directionbox{s3}{Follow the green line to the triangle.  Stop and face the centre.}

\directionbox{p}{Follow the pink line to the blue line to the podium, and stand behind it.}

\newpage

\subsection{Coda} % 1 minute

\playanote{}

\notewait{}

\speak{p}{Begin to close.}

\speak{s1}{Who wrote this?  What makes them authorities?}

\speak{s2}{Why should you pay such close attention to people reading words and following directions they have never seen before tonight?}

\speak{s3}{Who's the authority?  How is it being constructed?  What is this context here tonight?}

\speak{p}{Wrap it up.}

\speak{t1}{Authority.}

\speak{t2}{Constructed.}

\speak{t3}{Contextual.}

\technicalbox{q}{LX: \direction{Pause 2} A \(\downarrow\) 3 for three seconds, then A \(\uparrow\) 10.    Mirror ball motor and light on.}

\tsdirectionforall{The show is now over.  As people applaud, follow the next instructions to take your two bows.}

\newpage

\subsection{Curtain call}

\tsdirectionforall{Find your way to the plinth, where you will stand facing each other.}

\directionbox{p}{Stand on the orange chevrons, facing in.}

\directionbox{s1}{Stand on the orange square, facing in.}

\directionbox{s2}{Stand on the orange X, facing in.}

\directionbox{s3}{Stand on the orange triangle, facing in.}

\directionbox{musician}{Stand away from the piano, facing the centre.}

\directionbox{performers}{Bow once. \\
  \\
  S1, S2, S3: Turn half-way around and face out. \\
  \\
  All bow again, then follow the blue line to the exit and leave the Great Hall.}

\technicalbox{q}{LX: House lights to full.}

\newpage

\section{References}

% Cage and mushrooms

\textbf{``most extensive private library''} Tomkins 122;
\textbf{never ate magic mushrooms} Tomkins 123;
\textbf{Yoko Ono quote} Hoenigman;

% Music

\textbf{``gone too far''} Gann (2010) 191;
\textbf{Philip Glass quote} Glass 95--96;
\textbf{``one of the central musical figures of the mid-twentieth''} Gann (2002) 243;

\newpage

\section{Bibliography}

All web sites accessed 13 February 2020.

\begin{mybiblist}

% \item Rauschenberg quote about ``printer and the press'' \url{https://www.moma.org/audio/playlist/40/641}

\item Gann, Kyle F.  \textit{No Such Thing as Silence: John Cage's \textit{4' 33''}}.  New Haven: Yale University Press, 2010.

\item Gann, Kyle F. ``No Escape from Heaven: John Cage as Father Figure,'' in \textit{The Cambridge Companion to John Cage}. Cambridge UK: Cambridge University Press, 2002.
x
\item Hoenigman, David F.  ``Yoko Ono, forever a force for peace.'' \textit{Japan Times}, 07 November 2009. \url{https://www.japantimes.co.jp/community/2009/11/07/general/yoko-ono-forever-a-force-for-peace/}.

\item John Cage Society.  ``Indeterminacy.''  ``John Cage.'' \url{https://johncage.org/indeterminacy.html}.

\item Kostelanetz, Richard.  \textit{Conversing with Cage}.  2nd ed.  New York: Routledge, 2003.

% \item ``there's no such thing as silence.''  Richard Kostelanetz, \textit{Conversing with Cage}.  2nd ed.  New York: Routledge, 2003.  P. 65.

\item Pritchett, James, Laura Kuhn and Charles Hiroshi Garrett.  S.v. ``John Cage.''  \textit{Grove Music Online}. \url{https://www.oxfordmusiconline.com/grovemusic/}.

\item Tomkins, Calvin.  \textit{The Bride and the Bachelors: Five Masters of the Avant-Garde}.  Rev.\ ed.  Harmondsworth UK: Penguin, 1976.

% Also https://madelinex.com/2018/09/05/yoko-and-john-cage/

% \item O'Neill, John P., ed.  \textit{Barnett Newman: Selected Writings and Interviews}. New York: Alfred A. Knopf, 1990.

% \item Shiff, Richard, Carol C. Mancusi-Ungaro and Heidi Cols\-man-Frey\-berger. \textit{Barnett Newman: A Catalogue Raisonné}. New Haven CT: Yale University Press, 2004.

% \item \textit{Wikipedia}, s.v. ``Barnett Newman,'' \url{https://en.wikipedia.org/wiki/Barnett_Newman}.

\end{mybiblist}

\newpage

\section{Credits}

Created by Ashley Williamson and William Denton.

Theatre Science would not exist without the Arts and Letters Club of Toronto.  Thanks to everyone we've ever worked with in any stage production at the club.  We're grateful to the Stage Committee for giving us the time for these performances.  Special thanks to Fiona McKeown and Chris Gardener.  \textit{Very} special thanks to Michael Spence, who embodies both Theatre and Science.

Thanks to everyone who helped along the way:  Lisa Aikman, Carol Anderson, Ramona Baillie, William Blakeney, Sophie Bury, Sarah Coysh, Donald Gillies, Thomas Gough, Kris Joseph, Kathy Lennox, Damon Lum, Patti Ryan, Dany Savard, Betty Trott, Josh Welsh, Brenda Williamson.

Theatre Science is licensed under a Creative Commons Attribution 4.0 International license (CC BY).  % \ccby
