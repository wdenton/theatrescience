\section{Performers}

\begin{description}[align=right,labelwidth=2cm]

  \item [A] the audience
  \item [M] one musician
  \item [P] one primary speaker
  \item [S] three secondary roles
  \begin{description}[align=right,labelwidth=2cm]
    \item [S1] does one ``Mushroom Verbatim'' % can go up the stairs to the stage;
    \item [S2] does one ``Mushroom Verbatim'' % always on the floor;
    \item [S3] only speaks from the score % limited movement; stays in the podium zone
  \end{description}
  \item [T] three tertiary roles (on stage)
  \begin{description}[align=right,labelwidth=2cm]
    \item [T1] can sit or stand
    \item [T2] can sit or stand
    \item [T3] can sit or stand
  \end{description}
  \item [Q] technical person in the booth
  \item [R] technical person on the floor

\end{description}

P does about 45\% of the work, while the secondary group shares another 45\% and the tertiary group shares 10\%.

\newpage

\section{Instructions}

\subsection*{Formatting}

Boxes have \textit{stage directions}: instructions to be followed but not spoken.

  \setlength{\fboxrule}{2pt}%
  \vspace{0.3\baselineskip}%
  \fbox{\parbox{0.9\textwidth}{\textbf{A bold box has directions for you to follow.}}}%
  \setlength{\fboxrule}{1pt}%
  \vspace{0.3\baselineskip}

% \directionbox{}{A bold box has directions for you to follow.}

\vspace{0.5\baselineskip}
\fbox{\parbox{0.9\textwidth}{A regular box has directions for other people.  Do not perform these.}}
\vspace{0.5\baselineskip}

% \setlength{\fboxrule}{1pt}

\technicalbox{Q}{A dashed box has technical cues, which are also in italics to set them apart. Only Q and R will act on them.}

Your dialogue may include very brief directions.  These are shown with square brackets and small caps: \direction{Like this}.

Your dialogue is in bold.  Here is an example.

\begin{quote}

L:  The show tonight is in the Great Hall at the Arts and Letters Club.

\textbf{\performerletter{\speaker}:  \direction{Pause 1} We will be making use of the whole space, in and around you, but no audience member will be involved in the show.}

M:  The performance will last about forty minutes.

\end{quote}

\subsection*{Pauses}

\direction{Pause 1} means pause for a count of one, which you might time by saying to yourself, ``One one thousand.''.  \direction{Pause 2} means pause for a count of two, which you might time by saying to yourself, ``One one thousand, two one thousand.''  \direction{Pause 4}, you can imagine, will seem fairly long.

Don't rush a pause.  The audience will wait for you.

\subsection*{Pronunciation}

Hard-to-pronounce words also have a phonetic guide, where capital letters indicate emphasis:

\begin{quote}

  Composer Arnold \pronounce{Schoenberg}{SHERN-berg}.

  Cage was a mushroom expert: a \pronounce{mycologist}{my-COLL-o-jist}.

\end{quote}

Only say the word once.

If you are unsure of the pronunciation, give it your best shot, but don't worry if you make a mistake.  Try again or just move on.  If the audience noticed, they will forget quickly.

``Fungi'' can be pronounced in different ways.  We like to say it with a hard G so it sounds like ``funky.''

\subsection*{\textit{4' 33''}}

The John Cage composition \textit{4' 33''} should be pronounced in full as ``four minutes and thirty-three seconds.''

During the performance, don't worry if there is noise or laughter or things happen.  Just pay attention and listen.

\subsection*{Movement}

Movements at the start or end of a scenario are to be done by everyone simultaneously.

\subsection*{For the musician}

The piano indicates when a new scenario begins.  You will see these instructions:

\directionbox{m}{Wait until everyone has stopped moving, then play a note.}

To play a note, pick any key on the piano, hit it hard, and hold it down until you can no longer hear any sound, even if performers start talking.  Wait a few more seconds to be sure, then lift up your finger.

\newpageforperformers{}

\section{Setup}

% \subsection*{Requirements}

% \begin{itemize}

%   \item A piano or any other instrument that can make a lingering sound.
%   \item A projector and a screen, to show a PDF.
%   \item Headphones and a way of playing audio on them.

% \end{itemize}

% \newpageforperformers{}

\subsection*{Technical}

\technicalbox{r}{Projector and screen setup:  Open the red curtains about two-thirds open, and the greens about half.  Adjust them later so that the greens are as tight in as can be but everyone can see past the reds.\\
  \\
  Put the projector on a table against the back wall of the stage.  Put the laptop on the floor of the stage, extreme left, behind the podium, facing away from the audience.  Run the HDMI cable into HDMI 2 on the projector.  If needed, download the latest slides from the GitHub repository.  Change the projector settings so it is showing a reversed image.  Display the slide PDF full screen, and leave the first slide (``Williamson \& Denton Investigate Theatre Science'') up.  This will stay up through dinner.}

\technicalbox{q}{LX: House lights to Look 1 (full).  Work lights off.  Lighting board on.  GM\(\uparrow\)10. Single-scene mode.  Flash off.  A\(\uparrow\)10.  B\(\uparrow\)0.  (3,4)\(\uparrow\)4  42\(\uparrow\)10.  Great Hall: podium light on; buffet lights aimed at corners.}

\technicalbox{q}{SX:  Amp on.  Sound board on.  Master audio level to the mark.  Laptop headphone audio to ¾.  Plug laptop into 2 TK in the back.  2 TK input level to 5.  Prepare the headphone extension cord and the headphones, but leave them in the gallery.}

\technicalbox{q}{SX:  Prepare the microphone for the announcement.  Get a handheld microphone from the podium drawer and check that the battery is good enough.  Check which channel it's on, and move that input level to the mark.  Put it on mute.  Turn the microphone off.}

\technicalbox{q}{SX:  Prepare ``01 Thirty Minute Countdown.mp3'' and ``02 Introduction to Audience.mp3'' so they are ready to play one after the other without manual intervention.}

\technicalbox{q}{SX:  Confirm the piano is ready to be played.  Open the front part of the lid and put up the music desk.  Put the lid at half stick if that does not interfere with visibility.}

\technicalbox{q}{SX:  At 1928, lower the headphones down to R (standing in the yellow square), who will place them on the plinth.  Tie the cable to the railing so it is over the doorway.}

\technicalbox{q}{SX:  At 1930, play the thirty-minute introductory background.  First announcement is about ten seconds in.  Adjust volume as needed through the half hour.  It gets louder as it goes, and should be noticeably loud at the finish.\\
  \\
There's no going back now.}

\technicalbox{q}{SX:  Bring the gong and the mallet up to the gallery.  Place them near the sound board.}

\technicalbox{q}{SX:  At 1958, when the two-minute warning has played, lower the house lights to Look 4 (house lights and buffet lights to 65\%), but use Button \#2 (because the control box in the gallery is confused).}

\technicalbox{q}{SX:  Turn on the microphone (it is still muted).}

\newpageforperformers{}

\subsection*{Cast}

\directionbox{performers}{R will meet with you before the show and take you downstairs so you can get your costume (a dinner jacket or tailcoat).\\
  \\
A few minutes before the show starts R will get you in place for your entrance, either backstage or outside the Great Hall.}

\newpageforperformers{}

\section{Score}

\subsection{Scenario 0: Announcements} % 1 minute

\technicalbox{q}{SX:  At 2000, the introductory announcement plays.  It runs 80 seconds.}

\technicalbox{q}{SX:  Unmute the microphone.}

\technicalbox{q}{SX:  Announcement:  \\
  ``Ladies and gentleman, performing Theatre Science tonight are:\\
  Marianne Fedunkiw as Primary, \\
  Mark Osbaldeston as Secondary One, \\
  Vipin Sehgal as Secondary Two, \\
  Steve Vieira as Secondary Three, \\
  Peter Russell as Tertiary One, \\
  Kerstie Rickard as Tertiary Two, \\
  Patti Ryan as Tertiary Three, \\
  Sally Holton as Musician, \\
  William Denton as Q and Ashley Williamson as R. \\
  If you wish, you may follow along with the score provided.  \\
  Theatre Science has begun.''}

% Use \blankspace for blanks.

\technicalbox{q}{SX:  Quickly mute the microphone.}

\technicalbox{q}{SX:  Make an impressive sound with the gong.}

\newpageforperformers{}

\subsection{Scenario 1: Introduction} % 4 minutes

\technicalbox{q}{LX:  (2,5)\(\uparrow\)8. (20,21)\(\uparrow\)8.}

\directionbox{m}{Follow the red line until you are behind the piano, then sit on the piano stool.}

\directionbox{p}{Follow the yellow line.  Stop on the square.}

\directionbox{s1}{Follow the blue line towards the right and stop on the X.}

\directionbox{s2}{Follow the green line on the left and stop on the X, then face P.}

\directionbox{s3}{Follow the pink line and stop on the chevrons, then face P.}

\directionbox{tertiary}{Follow the orange line until you come to the chair marked with your number.  You may sit on it or stand in front of or behind it.  You may change positions, but only when the others are moving at the end of a scenario.  Make yourself comfortable.}

\playanote{}

\directionbox{p}{When you can no longer hear the note, wait one more second.  Slowly raise your arms over your head, then slowly lower them back to your sides.}

\directionbox{mst}{Copy P's arm movements as best you can.}

\speak{p}{This show is about a concept from library science:  \textit{Authority is constructed and contextual.} }

\technicalbox{r}{SLIDE: ``Authority is constructed and contextual.''}

\speak{p}{This comes from a field of librarianship called \textit{information literacy}.}

\speak{s1}{\textit{Literacy}, on its own, can be defined simply as being able to read and write.  But the world is a lot more complicated now than it was five thousand years ago.}

\speak{s2}{There are many different kinds of literacies.  The equivalent with numbers is \textit{numeracy}.  Being able to read
  and follow maps is a kind of literacy.  There is also \textit{digital literacy} and \textit{media literacy}.}

\speak{s3}{For librarians the key concept is \textit{information literacy}.  Here's a definition.  It's dense, but we'll break it down.}

\technicalbox{r}{SLIDE: ``Information literacy is the set of \ldots''}

\directionbox{mst}{Note that some people move while P speaks.}

\directionbox{s1}{Go back on the blue line and stop on the blue triangle.}

\directionbox{s2}{Follow the green line and stop on the green triangle.}

\directionbox{s3}{Turn to the right and follow the blue line, then stop on the star.}

\directionbox{p}{Read while the others move.}

\speak{p}{``Information literacy is the set of integrated abilities encompassing the reflective discovery of information, the understanding of how information is produced and valued, and the use of information in creating new knowledge and participating ethically in communities of learning.''  Citation.}

\speak{t1}{Association of College and Research Libraries.}

\speak{t2}{\textit{Framework for Information Literacy for Higher Education}.}

\speak{t3}{2015.}

\speak{p}{In other words, information literacy is the ability to do these with information:}

\speak{t1}{Find.}

\speak{t2}{Use.}

\speak{t3}{Understand.}

\speak{t1}{Evaluate.}

\speak{t2}{Integrate.}

\speak{t3}{Share.}

\speak{p}{They don't have to all be done at the same time, or in that order.}

\speak{s1}{Just reading the news requires you to understand, evaluate and integrate, but the news is coming at you all the time, you don't have to go out and find it.}

\speak{s2}{And it happens in different contexts.  Helping a kid with a homework assignment about trees is a lot easier than understanding changes to provincial policy about the Green Belt, but the steps are the same.}

\speak{s3}{Information resources don't have to be written in words.  They could be drawings, or on film or audio.  Sometimes they are people.  We turn to others for information and expertise.}

\directionbox{p}{Move along the yellow line to the X.}

\speak{p}{Librarians used to think of information literacy as being very much about using computers to do research.  And they had long lists of specific tasks that people should be able to do.  But as they taught information literacy to students in colleges and universities, and as the world changed, they found this approach wasn't the best for:}

\speak{t1}{Find.}

\speak{t2}{Use.}

\speak{t3}{Understand.}

\speak{t1}{Evaluate.}

\speak{t2}{Integrate.}

\speak{t3}{Share.}

\technicalbox{r}{SLIDE: The six frames.}

\speak{p}{So in 2016 a group wrote \textit{Framework for Information Literacy for Higher Education}.  It has six \textit{frames}.  These frames are \textit{threshold concepts}.  That's the idea that there is a change that happens when you go from a beginner's understanding of something to starting to see the big picture of how it all fits together, with all the nuances and subtleties. The six frames are:}

\speak{s1}{Authority is constructed and contextual.}

\speak{s2}{Information creation as a process.}

\speak{s3}{Information has value.}

\speak{s1}{Research as inquiry.}

\speak{s2}{Scholarship as conversation.}

\speak{s3}{Searching as strategic exploration.}

\speak{p}{We're only going to talk about the first one.}

\speak{s1}{Authority---}

\speak{s2}{---is constructed---}

\speak{s3}{---and contextual.}

\speak{p}{And to help us, we're going to use an example.}

\directionbox{p}{Walk back along yellow tape to the entrance, turn right, cross to the blue square, then stop and face centre.}

\directionbox{s1}{Follow the blue line and stop on the chevrons.  Face the stage.}

\directionbox{s2}{Follow the green line to the X.  Cross to the pink X.}

\directionbox{s3}{Follow the blue line and stop on the blue square.  Face the centre.}

\technicalbox{q}{SX:  Turn off the microphone and unmute the channel.  Turn the master volume to 0.  Unplug the audio cable from the laptop, and plug in the headphones.}

\newpageforperformers{}

\subsection{Scenario 2: John Cage} % 3 minutes

\playanote{}

\notewait{}

\speak{p}{An example.  An artist.  A musician.  John Cage.}

\technicalbox{r}{SLIDE: Photo of John Cage.}

\directionbox{mst}{Face the screen with your bodies.}

\speak{p}{\textit{Grove Music Online} is the largest and most respected reference source for music.  It says about Cage:}

\speak{t1}{``Born Los Angeles, September 5, 1912; died New York, August 12, 1992.  American composer. One of the leading figures of the postwar avant garde.''}

\speak{t2}{``The influence of his compositions, writings and personality has been felt by a wide range of composers around the world.''}

\speak{t3}{``He had a greater impact on music in the 20th century than any other American composer.''}

\directionbox{p}{Gesture at the secondary group to direct attention to them.}

\speak{p}{More, please.}

\speak{s1}{In his twenties he studied for two years under composer Arnold \pronounce{Schoenberg}{SHERN-berg}, who later said of Cage ``he's not a composer, but he's an inventor---of genius.''  Early success led him to Chicago, where he met artist Max Ernst and got another break.  Ernst invited him to come to New York and stay with him and his wife.  That wife was art collector Peggy Guggenheim.}

\directionbox{s1}{Turn one quarter to the right.}

\technicalbox{r}{SLIDE: Photo of Donald Gillies.}

\speak{s2}{In New York Cage met Marcel \pronounce{Duchamp}{doo-SHOMP}, who became a lifelong friend.  They often played chess.  In fact, there's a Club connection to Cage and Duchamp and chess. In 1968, our own Donald Gillies produced ``Reunion'' at Ryerson, where Cage and Duchamp played chess and the moves triggered sounds played by unseen musicians.  Our own Carol Anderson was in the audience.}

\technicalbox{r}{SLIDE: Photo of John Cage.}

\directionbox{s2}{Turn one-quarter to the right.}

\speak{s3}{Cage used chance to take himself out of the process of composing.  He would often roll dice to determine what would happen next.  He wrote music for percussion, piano, tape, voice, small ensembles, and other instruments.  One piece is written for twelve radios, each played by two people: one controls the tuning and the other the volume and timbre.  Most famously, he used silence.}

\directionbox{s3}{Turn one-quarter to the left.}

\speak{s1}{He was enormously talented and always curious.  He was also a kind man.  Younger composers appreciated the interest and generosity he showed even when they were doing work very different from his.}

\directionbox{s1}{Face the centre.}

\speak{s2}{He appeared in a surprising number of television game shows.  He was a Zen Buddhist.  He laughed a lot.  He organized the first Happenings.  He collaborated with Robert \pronounce{Rauschenberg}{ROWSH-en-burg}, Yoko Ono, and many, many others.  He was a mushroom expert: a \pronounce{mycologist}{my-COLL-o-jist}.}

\directionbox{s2}{Face the centre.}

\speak{s3}{Cage spent most of his life with Merce Cunningham, a dancer and choreographer.  Cage composed many scores for the Merce Cunningham Dance Company, and together they toured the world.}

\directionbox{s3}{Face the centre.}

\speak{t1}{Chance.}

\speak{t2}{Zen.}

\speak{t3}{Silence.}

\speak{t1}{Laughter.}

\speak{t2}{Mushrooms.}

\speak{t3}{Merce.}

\speak{p}{Cage is best known for his silent piece.  That's \textit{4' 33''}.}

\directionbox{p}{Follow the blue line back to the yellow star, then go to the yellow X.}

\speak{p}{This is the score.}

\directionbox{p}{Gesture overhead.}

\technicalbox{r}{SLIDE: Score of \textit{4' 33''}.}

\speak{p}{It has three movements.  Each is just the word \pronounce{\textit{tacet}}{TASS-it}.  This tells the musician not to play.  It was first performed by pianist David Tudor in 1952, in a concert space in the woods that was open to the outside.  He came on stage and sat at the piano.}

\directionbox{ps}{Look at the Ts on stage.}

\speak{t1}{In the first movement, he didn't play the piano.}

\speak{t2}{In the second movement, he didn't play the piano.}

\speak{t3}{In the third movement, he didn't play the piano.}

\speak{p}{After four minutes and thirty-three seconds of not playing, he stood up, and the piece was over.  That's the silent piece.}

\speak{s1}{Except \ldots}

\directionbox{s1}{Raise one finger over your head.}

\speak{s2}{\direction{Pause 1} \ldots it's not \ldots}

\directionbox{s2}{Shrug your shoulders.}

\speak{s3}{\direction{Pause 2} \ldots silent.}

\speak{s1}{\direction{Pause 3} Even though the pianist wasn't playing \ldots}

\directionbox{s1}{Lower your finger.}

\speak{s2}{\ldots the audience still heard sounds \ldots}

\speak{s3}{\ldots from themselves, and the room, and the whole environment.}

\speak{p}{There's a famous quote from Cage.  Citation first.}

\speak{t1}{John Cage.}

\speak{t2}{\textit{Silence}.}

\speak{t3}{Fiftieth anniversary edition, 2011.}

\speak{p}{The quote is: ``There is no such thing as silence.''}

\directionbox{st}{Give an exaggerated shrug and knowing wink to an audience member of your choice.}

\directionbox{s1}{Walk the blue line to the pink line, go towards the stage, and stop on the X.}

\directionbox{s2}{Cross to the blue line and stop on the blue X.}

\directionbox{s3}{Walk the blue line towards the entrance, then cross to the yellow line and walk to the X.}

\directionbox{p}{Cross to the pink line, turn towards the entrance, then walk on the blue line and stop on the chevrons.}

\newpageforperformers{}

\subsection{Scenario 3: Authority} % 5 minutes

\playanote{}

\notewait{}

\speak{p}{Back to the concept.}

\technicalbox{r}{SLIDE: ``Authority is constructed and contextual''.}

\speak{s1}{Authority---}

\speak{s2}{---is constructed---}

\speak{s3}{---and contextual.}

\speak{t1}{Authority.}

\speak{t2}{Constructed.}

\speak{t3}{Contextual.}

\speak{p}{This is the concept in question.  This is the \textit{frame} we are going to try to understand.  Let's look at the definition.}

\directionbox{performers}{Turn to look at the screen.}

\technicalbox{r}{SLIDE: ``Information resources reflect'' paragraph.}

\speak{p}{Please read me that big block of text on the screen.}

\speak{t1}{``Information resources reflect their creators’ expertise and credibility, and are evaluated based on the information need and the context in which the information will be used.''}

\speak{t2}{``Authority is constructed in that various communities may recognize different types of authority.''}

\speak{t3}{``It is contextual in that the information need may help to determine the level of authority required.''}

\speak{p}{Citation.}

\speak{t1}{Association of College and Research Libraries.}

\speak{t2}{\textit{Framework for Information Literacy for Higher Education}.}

\speak{t3}{2015.}

\directionbox{p}{Face the piano.}

\directionbox{s1}{Face the piano.}

\directionbox{s2}{Face the piano.}

\directionbox{s3}{Face the podium.}

\technicalbox{r}{SLIDE: ``Information resources reflect'' paragraph with first line bold.}

\speak{p}{First of all, what's an information resource?}

\speak{s1}{A book.  A web site.  A documentary.  A Facebook post.  An article in an academic journal.  A thread on Twitter.  A newspaper report.  A pamphlet your doctor gave you.  Could be a person!}

\speak{p}{How do they reflect their creators' expertise and credibility?}

\speak{s2}{The book was written by someone who's spent years on the topic, and it's published by a major publisher.  The web site has recipes tested by a cook you like.  The post on Facebook was made by someone who was actually at the event.}

\speak{s3}{The academic article is by a team of people at a research laboratory at a university.   The thread on Twitter is by a cabinet minister, about the government's response to an emergency.}

\speak{p}{And how you use these depends on what you need, when you need it, and how you'll use it.}

\speak{s1}{If I'm planning a vacation in a city I've never been to before, I'd go to some web sites, get some travel guides, and ask people I know if they've been there.  That's enough for me as a \textit{tourist}.  But if I was thinking about doing \textit{business} there, that's a whole different thing.  I need information about the economics, demographics, regulations and so on.}

\speak{s2}{Or say I had to write something about the Group of Seven.  If I want to check who the ninth and tenth members were, the Wikipedia entry is enough.  But if I'm studying Canadian art history at university, I need books by art historians and other experts.}

\speak{p}{Next part.}

\directionbox{performers}{Turn to look at the screen.}

\technicalbox{r}{SLIDE: ``Information resources reflect'' paragraph with second line bold: ``Authority is constructed \ldots''}

\speak{p}{Authority isn't absolute.  Different people can recognize different types of authority in different situations.}

\speak{s3}{People can select their own authorities, based on religion or politics or culture.  Some authorities have that position because
  of a respected professional role, like nurses.  Or maybe it's about power, like the principal in an elementary school.}

\speak{p}{And the last part.}

\technicalbox{r}{SLIDE: ``Information resources reflect'' paragraph with third line bold:  ``It is contextual in that \ldots''}

\speak{p}{If I want to know what A.J.~Casson was like as a person, someone who knew Cass---maybe a member here who used to have lunch with him, and visited his house to buy a painting---is one kind of authority.   But if I want to know about the influence the Group has had on Canadian art over the last century, that's a different context.  I'd want to talk to an art historian.  Who could be the same person.}

\technicalbox{r}{SLIDE: ``Authority is constructed and contextual''.}

\speak{p}{Two last things about this frame.  There are some things people who are \textit{getting better} with this concept will do, such as:}

\speak{s1}{They will be able to define different types of authority, for example: subject expertise, societal position or personal experience.}

\speak{s2}{They know indicators that help determine an authority's credibility.}

\speak{s3}{They know that there may be scholars who are widely acknowledged as authorities in an area, but are still challenged by other scholars.}

\speak{p}{And they know they can develop their own authority, and that doing so comes with responsibilities.  Also, there are ways that people will \textit{think and act}.}

\speak{s1}{ People who understand this frame will maintain an open mind. They will look for authoritative sources, and remember authority doesn't need to come from a university degree.}

\speak{s2}{They will be aware of who is saying someone is an authority, and why.}

\speak{s3}{They will be aware of their own attitudes and biases.}

\speak{p}{Now the beginning is in place.  We know about information literacy.}

\speak{t1}{Find.}

\speak{t2}{Use.}

\speak{t3}{Understand.}

\speak{t1}{Evaluate.}

\speak{t2}{Integrate.}

\speak{t3}{Share.}

\speak{p}{And we know about the frame.}

\speak{s1}{Authority---}

\speak{s2}{---is constructed---}

\speak{s3}{---and contextual.}

\speak{p}{Now we go back to our example.}

\directionbox{p}{Follow the blue line to the podium, then stand behind it, facing the centre.}

\directionbox{s1}{Walk the pink line to the pink chevrons.  Face the centre.}

\directionbox{s2}{Walk the blue line to the green line, then stop on the triangle.}

\directionbox{s3}{Cross to the pink line and stop on the X.}

\newpageforperformers{}

\subsection{Scenario 4: Music} % 3 minutes

\playanote{}

\notewait{}

\speak{p}{John Cage.}

\technicalbox{r}{SLIDE: Photo of John Cage.}

\speak{t1}{Chance.}

\speak{t2}{Zen.}

\speak{t3}{Silence.}

\speak{t1}{Laughter.}

\speak{t2}{Mushrooms.}

\speak{t3}{Merce.}

\speak{p}{We start with a few words about his music.  We saw the score of \textit{4' 33''}.  If you know one thing about Cage, you know he composed a work of music where there is no music.  That is not an easy concept.  Even his own mother was unsure.  After the premier, she said to one of Cage's friends, ``Don't you think that John has gone too far this time?''}

% \speak{s1}{He arrived at it through years of work.  There was a lot of thought and intent behind it.  Even though his music could sound wild and difficult, it was made with care and deliberation.  He knew what he was doing.  And through his life he kept changing what he was doing.}

% \speak{s2}{Cage began to study and compose music in the early 1930s.  Early on he studied under the composer Arnold Schoenberg, who was impressed with Cage's work.  However, harmony was important to Schoenberg but not to Cage, and he soon moved on.  One of his early percussion pieces, ``Third Construction,'' is still commonly performed by ensembles today.}

% \speak{s3}{He invented the prepared piano.  By putting screws and bolts and weather stripping between the strings of a piano, when the keys are hit the sound produced is very different from a normal piano.  Cage said it was ``a percussion ensemble under the control of a single player.''}

% \speak{p}{This all sounds rather long, and will be broken up differently.  There will be one or two quotes from Cage about his approach to sound and music.}

% hard work
% good luck
% enormous talent

% His work was appreciated and he was becoming known.  In Chicago he met the surrealist painter Max Ernst, who invited him to come to New York and stay with him and his wife, Peggy Guggenheim.   Cage and his wife moved there and he had more success.  In 1942 he began to work with choreographer Merce Cunningham, and in 1943 he left his wife for him.  He would compose music for Merce's dances for the rest of his life.

% It was in New York that Cage met Marcel Duchamp and other artists.  He would

% Marcel Duchamp.  Joseph Albers.  Buckminster Fuller.  Joseph Campbell.
% Willem de Kooning.  Robert Rauschenberg.
% Virgil Thomson.  Pierre Boulez.  Morton Feldman.
% Cage introduced Pierre Boulez to Aaron Copeland (in Paris)

% I Ching.

% 4' 33''

% Radio piece.

% John and Yoko.

% Brian Eno.

% Nono p. 160 Conversing. Critical of Cage's politics

\speak{s1}{John Cage made music for seventy years.  He began by composing for percussion.  In the late 1930s he invented the prepared piano, by putting bolts and wires and other objects on or between the actual strings.  This turns the piano into a one-person percussion ensemble.}

\speak{s2}{In the late 1940s he became interested in Eastern philosophies, and the \pronounce{\textit{I Ching}}{ee CHING} led him to use chance in his music.  He used this technique for the rest of his life.  He was opposed to improvisation, however, because it meant ``playing what you know.'' He wanted new musical experiences, and used unusual notation to help achieve that.}

\speak{p}{In the late 1950s he began to move into a new phase where he wanted to build \textit{indeterminacy}.}

\speak{s3}{He said, ``More essential than composing by means of chance operations, it seems to me now, is composing in such a way that what one does is indeterminate of its performance.   In such a case one can just work directly, for nothing one does gives rise to anything preconceived.''}

\speak{p}{For example, he might put marks on a piece of paper and lines on a clear transparent sheet.  Then a performer would put the transparent sheet on top of the paper, pointing whichever way they want, and that was the score.}

\speak{s1}{He sometimes made music that lasted for hours.  French composer Erik \pronounce{Satie}{sa-TEE} wrote a one-minute piece called \textit{Vexations}, and included cryptic instructions (always ignored) that mention playing it 840 times.  Cage took him at his word and ran a concert where pianists worked in shifts to play it 840 times, which took about sixteen hours.}

\speak{s2}{Cage's work \pronounce{\textit{Organ\textsuperscript{2} ASLSP}}{Organ 2 A--S--L--S--P} is currently being performed at a church in Germany.  The instructions say it should be performed ``as slow as possible.''  This performance started in 2000 and will last 639 years.   It begins with a rest, so for the first seventeen months no note was played.}

\speak{s3}{Cage also wrote and spoke about music.  He published many articles and several books.  His collected writings, particularly \textit{Silence} from 1961, are still frequently quoted.  He was an excellent interview subject, and was very open about his methods and philosophy.  All in all there are many volumes filled with his words.}

\speak{p}{You may be unfamiliar with Cage's music.  It's rare to hear it without making a special effort, though if you have access to a streaming music service you will find a lot there, with more added every year.  And remember \textit{Grove Music Online}: ``He had a greater impact on music in the 20th century than any other American composer.''  Next: mushrooms.}

\technicalbox{r}{SLIDE:  Mushrooms.}

\directionbox{s1}{Follow the pink line to the X.}

\directionbox{s2}{Follow the green line towards the stage.  Stop on the X.}

\directionbox{s3}{Follow the pink line towards the entrance curtains, then cross to the yellow square.  Face the centre.}

\directionbox{p}{Follow the blue line towards the entrance curtains.  Stop on the blue chevrons.}

\newpageforperformers{}

\subsection{Scenario 5: Mushrooms} % 5 minutes

\playanote{}

\notewait{}

\speak{p}{Cage became interested in mushrooms in 1954, when he was staying at a farmhouse.  He said:}

\speak{s1}{``I found myself living in small quarters with four other people, and I was not used to such lack of privacy, so I took to walking in the woods.  And since it was August, the fungi are the flora of the forest at that time.''}

\speak{s2}{``I was very involved with chance operations in music, and I thought it would just be a very good thing if I could get involved with something where I could not take chances.''}

\speak{p}{Asked if he had any favourite mushrooms, he said:}

\speak{s3}{``I like the ones I have.  If you like the ones you don't have, then you're not happy.''}

\speak{p}{Citation.}

\speak{t1}{Richard \pronounce{Kostelanetz}{KOS-tell-ANN-ets}.}

\speak{t2}{\textit{Conversing with Cage}.}

\speak{t3}{2003.}
% Kostelanetz, p. 16.

\speak{p}{Cage quickly became an expert, so much so that four years later he won a lot of money on Italian TV.  While in Milan, on a concert tour, he went on a game show called \textit{Leave or Double}.  Contestants won money by answering questions on a specialty subject.  His was mushrooms.  He appeared several times, building up his winnings.  Finally he had one-and-a-half million \textit{lire}, and in his last appearance he took the risk of answering three more questions to either win five million \textit{lire} (about eight thousand dollars) or lose it all.}

% https://johncagetrust.blogspot.com/2011/04/lascia-o-raddoppia-milan-1959.html
% CITE: https://johncage.org/blog/transcription.html
% CITE: BEGIN AGAIN, p. 168.

\speak{s1}{For the first question, he was shown photographs of mushrooms and had to identify them.  He got each one correct.}

\speak{s2}{Next he was shown a photo of a particular mushroom and had to give its scientific name, the colour and shape of its spores, and the length and width of its spores.  He gave all the right answers.}

\speak{s3}{Finally he was asked to list the twenty-four types of white-spore mushrooms given in a standard text.  Cage said, ``I can enumerate the list alphabetically,'' and then did so.}

\speak{p}{Cage later said, ``That was the first consequential amount of money I'd ever earned.  It was two years later, in 1960, that I began to make some money with my music.''  Peggy Guggenheim converted his winnings into American money on the black market.}

\directionbox{s1}{Walk straight to the yellow X in the centre, at the plinth.}

\directionbox{s2}{Walk straight to the yellow square in the centre.}

\directionbox{s3}{Walk to the steps.  Sit on the green X.}

\directionbox{p}{Move to the yellow star.}

\speak{p}{Mushrooms became a regular part of Cage's music and writing.  We are going to hear two stories from his 1959 work \textit{Indeterminacy}.}

\speak{s3}{My two colleagues are now going to perform ``Mushroom Verbatim.''  One after the other, they are going to put on those headphones and repeat what they hear.  It is not easy to repeat what you are hearing, without preparation.  We will forgive them any mistakes.  First, story number sixty-six.}

\directionbox{s1}{Stand in front of the chair.  To prepare, read the instructions below but do not follow them.  When you know what is going to happen, go through them again, but this time follow them. \\
  \\
  Instructions:  Put your binder down on the chair, face up so that you can see these words.  Put on the headphones.  Face the stage.  When you are ready, make a ``thumbs up'' sign with each hand.  You will hear three bongs, then a voice speaking.  Repeat what you hear.  To help your voice be heard, look at the top of the stage curtains while you do this.  You will hear three bongs when it's done.}

\directionbox{nots1}{While the Mushroom Verbatim is happening, look at the speaker.}

\technicalbox{q}{SX: At the thumbs up, play ``03 Indeterminacy 66.mp3'' over headphones.}

\directionbox{s1}{When the story is over, put the headphones back on the plinth and pick up your binder.  Cross to the green X.}

\speak{p}{Next, story number sixty.}

\directionbox{s2}{Cross to the plinth and stand in front of the chair.  To prepare, first read the instructions below but do not follow them.  When you know what is going to happen, go through them again, but this time follow them. \\
  \\
  Instructions:  Put your binder down on the chair, face up so that you can see these words.  Put on the headphones.  Face the stage.  When you are ready, make a ``thumbs up'' sign with each hand.  You will hear three bongs, then a voice speaking.  Repeat what you hear.  To help your voice be heard, look at the top of the stage curtains while you do this.  You will hear three bongs when it's done.}

\directionbox{nots2}{While the Mushroom Verbatim is happening, look at the speaker.}

\technicalbox{q}{SX: At the thumbs up, play ``04 Indeterminacy 60.mp3'' on the headphones.}

\directionbox{s2}{When the story is over, put the headphones back on the plinth, pick up your binder and return to the yellow square, then cross to the green tape and go towards the entrance curtains.  Turn onto the blue line and go to the blue square.  Face the piano.}

\directionbox{p}{When S2 is done and starts to move, walk on the yellow tape and stand on the X.  Face the piano.}

\directionbox{s1}{Follow the pink line towards the entrance to the blue line and walk to the chevrons.  Face the piano.}

\directionbox{s3}{Come off the steps.  Follow the blue line to the X.  Face the piano.}

\newpageforperformers{}

\subsection{Scenario 6: Cage and authority} % 5 minutes

\playanote{}

\notewait{}

\speak{p}{Cage was an expert on mushrooms, but he wasn't \textit{just} an expert.  He was also an \textit{authority}, which is different.  Someone can be an expert without being an \textit{authority}.}

\speak{t1}{Authority.}

\speak{t2}{Constructed.}

\speak{t3}{Contextual.}

\speak{s1}{He taught mushroom identification at a university in New York.  This led to him co-founding the New York \pronounce{Mycological}{my-co-LOJ-ical} Society, which is still operating today.}

\speak{s2}{He had ``perhaps the most extensive private library ever compiled on the subject'' of mushrooms.}

\speak{s3}{For a few months in 1960 he sold wild mushrooms to the Four Seasons restaurant in New York.}

\speak{p}{Cage was an \textit{authority} on mushrooms.  Definition again, please.}

\technicalbox{r}{SLIDE: Show ``Information resources reflect'' paragraph.}

\speak{t1}{``Information resources reflect their creators’ expertise and credibility, and are evaluated based on the information need and the context in which the information will be used.''}

\speak{s1}{Cage himself was an information resource.  We know about his expertise and his credibility.  If you wanted to know about mushrooms, perhaps while out in the forest, he would be a great person to ask.}

\speak{t2}{``Authority is constructed in that various communities may recognize different types of authority.''}

\speak{s2}{He was recognized by other mushroom lovers and researchers for his knowledge and his association work.}

\speak{t3}{``It is contextual in that the information need may help to determine the level of authority required.''}

\speak{s3}{If you needed a genetic analysis of a fungus, you'd go to a researcher in a lab.  But if you needed to identify a mushroom you'd found in the forest, he could do it.  He could tell you if it was poisonous or not.  You could trust John Cage to prevent you from accidentally dying from mushroom poisoning.}

\technicalbox{r}{SLIDE: ``Authority is constructed and contextual.''}

\speak{p}{Cage was an authority on mushrooms.  What about music?  Here's some evidence from conceptual artist and musician Yoko Ono, who met Cage in New York in the early sixties.  She said:  ``I did not immediately take to someone who was considered an authority of some kind.  I mean John Cage was already an authority in some ways, you know; people respected him.  I was just being a bit cynical.'' Cage's influence wasn't just from his compositions.  It also came from his writing, as composer Philip Glass relates.  He said:}

% \speak{p}{John Rockwell.}

% \speak{s1}{``Under Mr. Cage's benign influence, younger composers felt free to be free, to dispense with presuppositions about what music is and music be and to explore where their instincts and interests and modern technology took them.''}

% \speak{s2}{``The result has been a lot of trivial trash, of course, but there have also been some masterpiece---usually sounding totally unlike Mr.~Cage's own music, healthily enough---and even whole movements (Minimalism, most prominently) that owe their very existence to Mr.~Cage's liberating example.''}

% \speak{s3}{``Curiously, however, that example has been conveyed primarily through Mr.~Cage's writings (above all the seminal collection \textit{Silence} of 1961 \ldots) and the general example of his personality.''}

% \speak{p}{Citation.}

% \speak{t1}{John Rockwell.  Or something expressing the same idea.}

% \speak{t2}{``The Impact and Influence of John Cage'', in \textit{The New York Times}.}

% \speak{t3}{1987.}

% \speak{p}{That article was written when a new series of recordings of Cage's music was being released.  But it quotes Cage: ``I myself do not use records.  I just listen to the sounds around wherever I happen to be.'' \direction{Pause 2}  As noted, Cage's life and writings are more influential that his musical methods.  Here is a quote from the autobiography of composer Philip Glass.}


\speak{s1}{``The book \textit{Silence} was in my hands not long after it came out, and I would spend time with [two friends] talking and thinking about it.  As it turned out, it became a way that we could look at what Jasper Johns, Robert \pronounce{Rauschenberg}{ROWSH-en-burg}, Richard Serra or almost anybody from our generation or the generation before us did, and we could understand it in terms of how the work existed in the world.''}

\speak{s2}{``The accepted idea when I was growing up was that the late Beethoven quartets or \textit{The Art of the Fugue} or any of the great masterpieces had a platonic identity---that they had an actual, independent existence.  What Cage was saying is that there is no such thing as an independent existence.''}

\speak{s3}{``The music exists between you---the listener---and the object you're listening to.  The transaction of it coming into being happens through the effort you make in the presence of that work.  The cognitive activity is the content of the work.''}

\speak{p}{Citation.}

\speak{t1}{Philip Glass.}

\speak{t2}{\textit{Words Without Music}.}

\speak{t3}{2015.}

\speak{p}{Cage influenced minimalism, conceptual art, Fluxus and other movements.  There are many people who admire him, but there are even more who don't.  To them his work is somewhere between an interesting curiosity and an absurdity that has nothing to do with actual music.}

\speak{s1}{Composers who also did new and sometimes difficult music didn't always agree with him.  Pierre \pronounce{Boulez}{boo-LEZZ} and Karl-Heinz \pronounce{Stockhausen}{SHTOCK-how-sen} thought his use of chance was misguided.  Luigi \pronounce{Nono}{NO-no} said his work was ``profoundly reactionary'' (because it did not allow true creative liberty) and called Cage's followers ``products of a narcissistically coquettish pseudo-radicalism.''}

\speak{s2}{YouTube has many reactions to \textit{4' 33''}.  A few examples.}

\speak{t1}{``Finally I've found a song I can master!''} % Nemo TheEight

\speak{t2}{``I'm setting this as my ringtone.''} % tenphan0n0

\speak{t3}{``3:27 is the best part.''} % skybluejellybeans

\speak{t1}{``Good tempo chosen by the conductor.''} % seadog365

\speak{t2}{``This is not funny, it is the direct consequence of deconstructionism (extreme liberalism).''} % claude lara

\speak{t3}{``This is probably one of the stupidest things I have ever heard, not heard?  My music history professor has never been angrier than when she was teaching us about John Cage and this stupid song.''} % Captian Bigums

\speak{s3}{Our own Betty Trott, professor emerita of philosophy at Ryerson, taught the silent piece in a course on the aesthetics of music.  She said, ``The aesthetics explanation was an attempt to focus on the silence, or `not music,' as part of composition.  Most students grasped the principle, but dismissed the result.
% with the question, `Why is four minutes of audience noise of any interest at all?'
\ldots Why, my students said, would you pay money to listen to other people sneeze?  Good question, I thought. Sometimes theory is best left on the page.''}

\speak{p}{Composer and critic Kyle Gann sums it up:}

% \speak{s1}{``How to explain the fact that, for hundreds of composers, Cage remains one of the central musical figures of the mid-twentieth century, if not \textit{the} central one? ''}

\speak{s1}{``For thousands trained in American conservatories, Cage is merely a historical irritant, a charlatan against whom their teachers warned them.  For another type, Cage is a model, a basic paradigm for a composer's life.''}

\speak{s2}{``In America, this is probably the most reliable distinction between what is called the Uptown composer (Uptown as in Manhattan) and the Downtown composer.''}

\speak{s3}{``Uptown composers mention Cage with anger or contempt or condescension; Downtown composers talk of him with reverence, respect, and seriousness.  Perhaps no other composer in history elicits such extremes of approval and disapproval.''}

\speak{p}{Citation.}

\speak{t1}{Kyle Gann.}

\speak{t2}{``No Escape from Heaven: John Cage as Father Figure,'' in \textit{The Cambridge Companion to John Cage}.}

\speak{t3}{2002.}

\speak{p}{Cage's authority on mushrooms seems clear.  Whether or not he is an authority on music is entirely different.  Is he an authority for \textit{you}? Do you look to him for lessons about how to find and create art and life?  Perhaps he \textit{isn't}: nothing of his work is an influence, and he and his followers are completely removed from your practice.}

\speak{s1}{In which case, you will have different authorities.  Either way, the authority is constructed.  Different communities recognize different types of authority.  Here in Toronto, Cage is an authority at Soundstreams and the Music Gallery.  At Opera Atelier and Tafelmusik, he is not.}

\speak{s2}{And it's contextual.  Do you want to let ambient environmental sounds to be part of a performance?  That will lead you to Cage.  If you want to perform a crowd-pleasing Mozart symphony, you will look elsewhere.}

\speak{s3}{You can make up your own mind about whether Cage's artistic expertise is useful for what you need.}

% Doesn't have to be about art and music.
% Perform any disciplined action.
% Quote about attention?

\speak{p}{What if you knew very little about John Cage before tonight?  Maybe you're unsure what to make of all this.  We're going to give you some time to think about it.  And experience it.  We're going to perform \textit{4' 33''}.}

\technicalbox{r}{SLIDE: \textit{4' 33''} score.}

\directionbox{s1}{Walk directly towards the stage and sit on the chair with a green triangle.}

\directionbox{s2}{Walk directly towards the stage and sit on the chair with a green square.}

\directionbox{s3}{Walk back towards the stage and sit on the steps on the green X.}

\directionbox{tertiary}{Sit in your chair.}

\directionbox{p}{Remain standing in front of the plinth.}

\newpageforperformers{}

\subsection{Scenario 7: \textit{4' 33''}} % 6 minutes

\playanote{}

\notewait{}

\speak{p}{I will conduct.}

\speak{s1}{The performers will be three of us, and the three of them on stage, and the musician at the piano.  None of us have rehearsed this.}

\speak{s2}{The conductor will put on headphones, but this is \textit{not} a Mushroom Verbatim.  The conductor will listen to but \textit{not} repeat directions on how to conduct us.}

\speak{s3}{We are going to read the instructions in our scores, and then follow them.  As you know, there are three movements.}

\speak{t1}{In the first movement, we won't play an instrument.}

\speak{t2}{In the second movement, we won't play an instrument.}

\speak{t3}{In the third movement, we won't play an instrument.}

\speak{p}{There will be brief pauses between the movements.  We will now prepare.}

\directionbox{mst}{To prepare, first read the instructions below but do not follow them.  When you know what is going to happen, go through them again, but this time follow them. \\
\\
  Instructions: You are going to pay close attention to P, who will conduct.  Do not do anything.  Just pay attention and listen.  When the piece is over, P will say, ``That was \textit{4' 33''}.''  Then you will stand and bow with the conductor.  \\
  \\
  When you are ready, place your score face up on the floor in front of you (the musician can leave the score where it is).  Then pay attention.}

\directionbox{p}{To prepare, first read the instructions below but do not follow them.  When you know what is going to happen, go through them again, but this time follow them.\\
  \\
  Instructions: This is \textit{not} a Mushroom Verbatim.  \textit{Do not repeat what you hear}.\\
  \\
  Stand facing the stage.  Place your score face up on the chair.  Put on the headphones.  When the others have put their scores on the floor, they are ready.  When you are ready, make a ``thumbs up'' with each hand.  You will hear three bongs, then Ashley Williamson telling you what to do.  Follow the instructions.  You will end by saying, ``That was \textit{4' 33''}.''}

\technicalbox{q}{SX: At the thumbs up, play ``05 433 Instructions.mp3'' on the headphones.}

\directionbox{mst}{After P has said ``That was \textit{4' 33''},'' and you have stood and bowed with P, pick up your scores and follow the next directions.}

\directionbox{s1}{Go to the pink chevrons, then follow the blue line to the star.  Face the centre.}

\directionbox{s2}{Follow the blue line to the triangle.}

\directionbox{s3}{Follow the green line to the triangle.  Stop and face the centre.}

\directionbox{p}{Follow the pink line towards the stage to the blue line, walk to the podium, and stand behind it.}

\newpageforperformers{}

\subsection{Coda} % 1 minute

\playanote{}

\notewait{}

\technicalbox{r}{SLIDE: ``Theatre Science is constructed and contextual.''}

\speak{p}{What about this show?  Who's the authority \textit{here}?  Is it me, standing here saying these words I didn't write?}

\speak{s1}{Why should you believe someone who's reading things aloud from a binder they received an hour ago?}

\speak{s2}{Williamson and Denton present themselves as authorities.   Are they?  They appear to have some credentials.  Ashley Williamson has a PhD in theatre studies.  William Denton is a librarian at York University.}

\speak{s3}{And this show combines their expertise: theatre methods with library science content.  They cited their work.  There's a bibliography.  The core concept comes from a reputable source.}

\speak{p}{We leave it to you.  You've had the background.}

\speak{t1}{Find.}

\speak{t2}{Use.}

\speak{t3}{Understand.}

\speak{t1}{Evaluate.}

\speak{t2}{Integrate.}

\speak{t3}{Share.}

\speak{p}{You heard the example.}

\speak{t1}{Chance.}

\speak{t2}{Zen.}

\speak{t3}{Silence.}

\speak{t1}{Laughter.}

\speak{t2}{Mushrooms.}

\speak{t3}{Merce.}

\speak{p}{Who's an authority?  When?  Why?  Are \textit{they}?  Are \textit{you}?}

\speak{t1}{Authority.}

\speak{t2}{Constructed.}

\speak{t3}{Contextual.}

\technicalbox{q}{LX: \direction{Pause 2} A\(\downarrow\)3 for three seconds, then A\(\uparrow\)10, then (7,8)\(\uparrow\)10 (mirror ball).}

\directionbox{performers}{The show is now over.  As people applaud, follow the next instructions to take your bow.}

\newpageforperformers{}

\subsection{Curtain call}

\directionbox{ps}{Find your way to the centre.}

\directionbox{p}{Stand on the orange chevrons, facing out.}

\directionbox{s1}{Stand on the orange square, facing out.}

\directionbox{s2}{Stand on the orange X, facing out.}

\directionbox{s3}{Stand on the orange triangle, facing out.}

\directionbox{m}{Stand away from the piano, facing the centre.}

\directionbox{tertiary}{Stand (or remained seated if you prefer).}

\directionbox{performers}{Bow. \\
  \\
  M, P and Ss: Follow the blue line to the exit and leave the Great Hall.\\
  \\
  Ts: Follow the orange line back off the stage.}

\technicalbox{q}{LX:  House lights to full.}

\technicalbox{q}{SX:  Put back the microphone.  Turn off the amp and sound board.}

\technicalbox{R}{Strike the projector and laptop.}

\newpageforperformers{}

\section{Citations}

\textsc{introduction}
\textbf{definition of information literacy} and
\textbf{\textit{Framework}} ACRL.

\textsc{john cage}
\textbf{Schoenberg quote} Kostelanetz 6;
\textbf{\textit{Grove Music Online}} Pritchett, Kuhn and Garrett;
\textbf{various aspects of his life} Silverman and Gann (2010);
\textbf{``no such thing as silence''} Cage (2011) 51.
% \textbf{``no such thing as silence''} Kostelanetz 70.

\textsc{authority}
\textbf{\textit{Framework for Information Literacy for Higher Education}} ACRL.

\textsc{mushrooms}
\textbf{farmhouse story and favourites} Kostelanetz 16;
\textbf{game show} Silverman 166--169;
\textbf{the Mushroom Verbatims} are stories 66 and 60 from Cage (2009).

\textsc{music}
\textbf{``gone too far''} Gann (2010) 191;
\textbf{``playing what you know''} Gann (2010) 282;
\textbf{``more essential than''} Cage (2011) 69;
\textbf{ASLSP} BBC;
\textbf{``greater impact on music''} Pritchett, Kuhn and Garrett.

\textsc{cage and authority}
\textbf{``private library''} Tomkins 122;
% \textbf{never ate magic mushrooms} Tomkins 123;
\textbf{Yoko Ono quote} Hoenigman;
\textbf{Philip Glass quote} Glass 95--96;
\textbf{Luigi Nono quote} Silverman 160;
\textbf{first YouTube three comments} are by Nemo TheEight, tenphan0n0 and skybluejellybeans on Tudor;
\textbf{second three comments} are by seadog365, claude lara and Captian Bigums on EBU Euradio Orchestra;
\textbf{Betty Trott quote} personal communication;
\textbf{``one of the central musical figures''} Gann (2002) 243.

\newpageforperformers{}

\section{Bibliography}

\begin{mybiblist}

\small

\item Association of College and Research Libraries (ACRL).  \textit{Framework for Information Literacy for Higher Education}, 2016. \url{http://www.ala.org/acrl/standards/ilframework}.

\item Baer, Andrea. ``It's all relative? Post-truth rhetoric, relativism, and teaching on `Authority as [\textit{sic}] Constructed and Contextual'.'' \textit{College \& Research Libraries News} 79 no.\ 2 (February 2018): 72--75, 97.  \url{https://crln.acrl.org/index.php/crlnews/article/view/16877/18515}.

\item BBC News.  ``First notes for 639-year composition.''  \textit{BBC News}, 05 February 2003.  \url{http://news.bbc.co.uk/2/hi/entertainment/2728595.stm}.

\item Bravender, Patricia, Hazel McClure and Gayle Schaub.  \textit{Teaching Information Literacy Threshold Concepts: Lesson Plans for Librarians}.  Chicago: Association of College and Research Libraries, 2015.

\item Cage, John.  \textit{Indeterminacy}.  Performance edition.  Edited by Eddie Kohler.  C.F. Peters No.~68142, 2009.

\item ---------  \textit{Silence}.  50th anniversary ed. Edited by Kyle Gann.  Middletown CT: Wesleyan University Press, 2011.

\item EBU Euradio Orchestra. \textit{4' 33''}.  31 May 2020.  YouTube video, 4:47.  \url{https://www.youtube.com/watch?v=OovYr0w7BMA}.

\item Gann, Kyle F. ``No Escape from Heaven: John Cage as Father Figure,'' in \textit{The Cambridge Companion to John Cage}. Cambridge: Cambridge University Press, 2002.

\item ---------  \textit{No Such Thing as Silence: John Cage's \textit{4' 33''}}.  New Haven: Yale University Press, 2010.

\item Glass, Philip.  \textit{Words Without Music}.  New York: Liveright, 2015.

\item Hoenigman, David F.  ``Yoko Ono, forever a force for peace.'' \textit{Japan Times}, 07 November 2009. \url{https://www.japantimes.co.jp/community/2009/11/07/general/yoko-ono-forever-a-force-for-peace/}.

\item Kostelanetz, Richard.  \textit{Conversing with Cage}.  2nd ed.  New York: Routledge, 2003.

% \item ``there's no such thing as silence.''  Richard Kostelanetz, \textit{Conversing with Cage}.  2nd ed.  New York: Routledge, 2003.  P. 65.

\item Pritchett, James, Laura Kuhn and Charles Hiroshi Garrett.  S.v. ``John Cage.''  \textit{Grove Music Online}. \url{https://www.oxfordmusiconline.com/grovemusic/}.

\item Ross, Alex.  ``Searching for silence,'' \textit{New Yorker}, 04 October 2010.  \url{https://www.newyorker.com/magazine/2010/10/04/searching-for-silence}.

\item Silverman, Kenneth. \textit{Begin Again: A Biography of John Cage}. New York: Alfred A Knopf, 2010.

\item Tomkins, Calvin.  \textit{The Bride and the Bachelors: Five Masters of the Avant-Garde}.  Rev.\ ed.  Harmondsworth UK: Penguin, 1976.

\item Tudor, David. \textit{4' 33''}.  n.d.  YouTube video, 5:48.  \url{https://www.youtube.com/watch?v=HypmW4Yd7SY}.

\item Wilson, Patrick.  \textit{Second-Hand Knowledge: An Inquiry Into Cognitive Authority}.  Westport CT: Greenwood, 1983.

\end{mybiblist}

All web sites accessed 18 February 2020.

The photograph of Donald Gillies is © 2019 William Blakeney, who provided it.  The photograph of John Cage is CC BY and from WikiArt.org.  The photograph of the \textit{Entoloma sinuatum} (not \textit{grayanum}, which are mentioned in the verbatim) is CC BY-SA by Archenzo.  The last two can be found in the relevant Wikipedia entries.

% Also https://madelinex.com/2018/09/05/yoko-and-john-cage/
% \item Rauschenberg quote about ``printer and the press'' \url{https://www.moma.org/audio/playlist/40/641}

\newpageforperformers{}

\section{Credits}

Created by Ashley Williamson and William Denton.

Theatre Science would not exist without the Arts and Letters Club of Toronto.  Thanks to everyone we've ever worked with in any stage production at the club.  We're grateful to the Stage Committee for giving us the time for these performances. Special thanks to Fiona McKeown, Chris Gardiner and Matthew Percy.  \textit{Very} special thanks to Michael Spence, who embodies both Theatre and Science.

Printing of the audience scores is funded by a grant from York University Libraries.

Thanks to everyone who helped along the way:  Lisa Aikman, Carol Anderson, Ramona Baillie, John Beckwith, William Blakeney, Sophie Bury, Sarah Coysh, Donald Gillies, Thomas Gough, Kris Joseph, Kathy Lennox, Rob Prince, Damon Lum, John Rammell (whom blizzards cannot stop), Kerstie Rickard, Patti Ryan, Dany Savard, Betty Trott, Josh Welsh, Brenda Williamson.

Performers on 20 February 2020: Thomas Gough as Primary, Michelle Hogan-Walker as Secondary One, Lorna Kelly as Secondary Two, John Rammell as Secondary Three, Irene Katzela as Tertiary One, Damon Lum as Tertiary Two, Lucy Brennan as Tertiary Three and Rob Prince as Musician.

Theatre Science is licensed under a Creative Commons Attribution 4.0 International license (CC BY).  % \ccby

\vfill

{\small Printed \DTMnow.}
